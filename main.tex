\documentclass{article}
\usepackage[T1]{fontenc}
\usepackage[utf8]{inputenc}
\usepackage[norsk]{babel}
\usepackage[a4paper, margin=3cm]{geometry}
\usepackage{siunitx}
\usepackage{hyperref}
\usepackage{cleveref}
\usepackage{xcolor}

\title{En ekstern arbeidsgruppes evaluering av NTNUs studieprogram MTFYMA og BFY}
\author{Arbeidsgruppa består av: \\ Dag Kristian Dysthe (UiO), \\ Sverre Gullikstad Johnsen (SINTEF), \\ Morten Hjorth-Jensen (UiO), \\ Karl Yngve Lervåg (SINTEF), \\ Kari Schjølberg-Henriksen (GE Healthcare), \\ Ingve Simonsen (NTNU), \\ Paal Skjetne (SINTEF)}
\date{\today}

\newcommand{\crefrangeconjunction}{--}
\crefname{section}{seksjon}{seksjoner}
\crefname{figure}{figur}{figurer}

\begin{document}

\maketitle
\tableofcontents

\section{Introduksjon}
Eksperiment, teori og beregninger er de sentrale bærebjelkene i dagens fysikkfag. Samspillet mellom  eksperiment, beregninger og teori har lagt grunnlaget for flere av våre innsikter om  viktige naturvitenskapelige prosesser  samt vår grunnlegende forståelse av sentrale naturlover og utvikling av vårt moderne teknologiske samfunn.
En grunnleggende forståelse av eksperiment, teori og beregninger er essensielle element i teknologisk innovasjon og utvikling, og legger grunnlaget for vårt framtidige kunnskapsbaserte samfunn. 

Numeriske modeller og beregninger spiller en sentral rolle i dagens naturvitenskapelige og teknologiske utvikling. Kompliserte eksperimenter erstattes mer og mer av numeriske beregninger. Kunnskap om og innsikt og evne til å behandle og analysere store datasett spiller også en sentral rolle. Mange viktige oppdagelser og teknologiske nyvinninger ville ikke vært mulig uten omfattende numerisk modellering og dataanalyse.  

Vitenskapelige beregninger har som mål å utvikle nøyaktige metoder og modeller som setter oss i stand til å studere komplekse naturvitenskapelige og teknologiske system og fenomen. Mange av disse er såpass komplekse at fysiske eksperiment fort blir for kostbare eller nesten umulige å gjennomføre i et laboratorium. Numeriske beregninger og modellering spiller dermed en sentral rolle i moderne vitenskap og teknologisk utvikling. 
Digital kompetanse, innsikter og ferdigheter bør dermed være naturlige element i et naturvitenskapelig studieprogram. For et eksperimentelt fag som fysikk, åpner det seg også mange nye pedagogiske muligheter i integrasjonen av numeriske beregninger, dataanalyse og eksperiment. Mange eksperiment kan nå gjennomføres ved hjelp av f.eks.\ enkle applikasjoner i smarttelefoner~(aksellerometer, {\color{red} REF}) og enkle apparat som Arduino~({\color{red} REF}). Dette åpner for en tettere integrasjon av beregninger, teori, dataanalyse og eksperiment i utdanningsløpet. På sikt muliggjør dette en dypere forståelse av fysikkfagets egenart og den vitenskapelige metode på et tidligere stadium i studieløpet.

Denne arbeidsgruppa er bedt spesielt om å se på følgende punkter:
\begin{itemize}
  \item I hvilken grad imøtekommer utdanningene samfunnets behov innenfor beregningsorientert og eksperimentell fysikk.
  \item Hvordan sammenlikner den eksisterende og den planlagte utdanningen med hvordan dette adresseres på andre læresteder (både i Norge og internasjonalt).

  {\color{red} KarlYngve: Hva mener vi med dette, setningen gir ikke mening.}
  \item Er det noe spesielt som mangler innenfor disse områdene i et bærekraftperspektiv?

  {\color{red} KarlYngve: Hva legges i ``disse områdene'', og hva er man egentlig ute etter med dette spørsmålet?}
\end{itemize}

\subsection{Rapportens relevans for studentene ved MTFYMA}
Medlemmene i den eksterne arbeidsgruppen har ikke oversikt over hvor studentene fra MTFYMA begynner å jobbe: hvilke arbeidsgivere er mest aktuelle for dem?
Vurderingene og anbefalingene i denne rapporten er basert på den eksterne arbeidsgruppens kunnskap om behov i forskjellige sektorer og bransjer.
Vi håper disse er relevante for mange av de arbeidgiverne som studentene fra MTFYMA kommer til å ha.

Medlemmene i den eksterne arbeidsgruppen har veiledet nyansatte, samt veiledet studenter på både bachelor-, master- og PhD-nivå.
Medlemmenes sektortilknytning over tid:

% Jeg har lagt inn min egen arbeidstid i forskjellige sektorer. Jeg foreslår å ta med dette for at studieledelsen kan se på våre anbefalinger samtidig som de tar hensyn til hvor studentene ofte begynner å jobbe. Hvis dere er enige, kan dere legge til deres egne år og sektorer. Det blir sikkert mange hundre år til sammen (Kari)

\begin{itemize}
  \item Akademia (9 år)
  \item Oppdragsforskning med nær kontakt med industri (25 år)
  \item Industri (2 år)
\end{itemize}

{\color{red} KarlYngve: Dette må forklares litt nærmere om vi har det med; det er ikke åpenbart at årene i parentes her er hvor mange år medlemmene totalt har vært involvert i de nevnte sektorene. Paal: enig, jeg kan legge beslag på 22.5 år av de 25 under oppdragsforskning ... :-0 }

\subsection{Rapportens struktur}



\section{Generiske ferdigheter, behov i framtidas samfunn}
\label{Behov}
Arbeidgivere vil forvente og vil ha nytte av at uteksaminerte studenter fra MTFYMA har faglig kunnskap innenfor fysikk, kjemi, matematikk, statistikk og programmering.
Arbeidgivere vil forvente at studentene har en solid teoretisk kunnskapsbase og at de kan gjøre avanserte beregninger, både analytiske og numeriske.

{\color{red} KarlYngve: Burde denne seksjonen kommet som nummer 2, som en slags innledningen til våre observasjoner og etterhvert vurderinger? Det virker fornuftig for meg.}

\subsection{Sammenfatte resultater av en undersøkelse i en rapport}
I arbeidslivet er det svært nyttig å kunne skrive en vitenskapelig rapport.
Strukturen i en vitenskapelig rapport er egnet til å dokumentere en prosess på en tydelig og ryddig måte som gjør det mulig å gjenskape hva som ble gjort og vurdert.
Den vitenskapelige rapporten har som formål å gjøre det tydelig for leserne hvilket problem som er behandlet, hvilken metode som er brukt til å behandle problemet og hvilket resultat metoden ga.
Videre skal rapporten gi en vurdering av resultatet og en konklusjon eller anbefaling.

Den vitenskapelig rapporten er mye brukt i industri og oppdragsforskning, selv om resultatene ikke nødvendigvis skal bli en vitenskapelig artikkel.
Strukturen i en vitenskapelig rapport er godt egnet som skriftlig dokumentasjon i mange sammenhenger, for eksempel til å presentere valg av leverandør for et innkjøp, valg av arkitektur for en IT-applikasjon, valg av teknisk løsning for et problem i industriproduksjon, resultater av et vitenskapelig eksperiment.
De formelle kravene til en vitenskapelig rapport er gode og nyttige verktøy for å sikre lesbarhet og sporbarhet i argumentasjonen.
Noen eksempler på formelle krav er å ha figurtekst under figurer, tabelltekst over tabeller, å referere til alle figurer og tabeller i teksten, og å skrive referanseliste riktig.

En vitenskapelig rapport må også diskutere og tolke resultatene, og sammenlikne dem med eventuelle resultater som er oppnådd av andre.
Dette plasserer undersøkelsen/eksperimentet i sammenheng med allerede etablert kunnskap.
Dette er igjen viktig for å vurdere resultatenes legitimitet.
Å kunne skrive en vitenskapelig rapport er derfor en generisk ferdighet som er etterspurt i arbeidslivet nesten uavhengig av fagfelt og arbeidsoppgaver.


\subsection{Erfaring med å jobbe både i prosjektform og selvstendig}
Mange oppgaver i arbeidslivet er organisert som et prosjekt med en prosjektleder og prosjektmedarbeidere.
Dette gjelder også eksperimentelt arbeid.
Det er nyttig for arbeidslivet at studentene får noe erfaring med å utføre eksperimentelt arbeid sammen med andre i prosjektform.
Når et eksperiment er organisert som et prosjekt med flere medarbeidere settes det flere krav til dokumentasjonspraksis hos hver enkelt medarbeider.

Samtidig som prosjektet er en vanlig arbeidsform er arbeidsgivere også avhengig av at arbeidstakerne kan jobbe selvstendig.
Evne til selvstendig arbeid kan bety å ha strategier for å kunne angripe en kompleks problemstilling på en konstruktiv måte.

\subsection{Formulere et forskningsspørsmål eller en testbar hypotese}
Motivasjonen for et eksperiment kan ha forskjellige kilder, for eksempel:
\begin{itemize}
  \item Å teste en ny teori.
  \item Å ha en kunde som betaler.
  \item Å velge en optimal teknologi for noe som skal utvikles.
  \item Å finne en løsning på et problem som har oppstått.
\end{itemize}

Hvilken motivasjon som er vanlig kan variere mellom sektorer og fagfelt.
I mange situasjoner der behovet er å velge optimal teknologi eller løse et eksisterende problem vil det være mest kostnadseffektivt å gjøre eksperimenter fremfor å utvikle nye teorier.
Det kan kreve store ressurser å utvikle teori for komplekse situasjoner som for eksempel fukting på overflater med varierende ruhet eller spredning av lys på en samling partikler med ujevn form og varierende størrelse.
Eksperimentelle resultater kan gi den nødvendige innsikten direkte.

Når eksperimenter skal brukes som grunnlag for å velge teknologi eller løse eksisterende problem er det essensielt at eksperimentet bygger på en klart formulert problemstilling (forskningsspørsmål) eller en testbar hypotese. Uten et klart forskningsspørsmål er det usikkert hvorfor eksperimentet skal gjennomføres. I arbeidslivet er det vanligvis bare eksperimenter med en klar nytteverdi som blir gjennomført. Derfor er det nyttig for arbeidslivet at arbeidtakerne har trening i å omsette et komplekst spørsmål i en eller flere forskningsspørsmål eller klart formulerte hypoteser.

\subsection{Kunnskap om god dokumentasjonspraksis}
I en jobbsituasjon må eksperimentelt arbeid dokumenteres. Det er nyttig for arbeidslivet at arbeidstakerne har kjennskap til retningslinjer for god dokumentasjonspraksis (GDP), god laboratoriepraksis (GLP), og dataintegritet (DI) og vet hvor de kan finne oppdateringer av disse retningslinjene.

Studenter bør være kjente med LIMS systemer og hva deres formål er og ikke er. 

\subsection{Kunnskap om god kodeutviklingspraksis}
I et stadig mer digitalisert næringsliv bør studenter være kjent med verktøy for versjonskontroll av kildekode og rutiner for testing (enhetstesting) og godkjenning av ny eller forbedret kode. Hvorvidt studenter bør være kjent med elementer av sikker kodedesign samt tungregning bør vurderes. I distribuerte kodeutviklingsprosjekter (i tid eller rom) bør studentene være kjente med ulike  arbeidsformer som benyttes i industrien, samt metoder for ``automatisk'' dokumentasjon av kode (f.eks. \verb+doxygen+).

\subsection{Kjennskap til risikoanalyse}
Mange arbeidsplasser og bedrifter bruker standardiserte metoder for risikoanalyse, f.eks. Failure Mode and Effect Analysis (FMEA). Det er nyttig for arbeidsgivere at nye ansatte har kjennskap til konseptet risikoanalyse, og gjerne kjennskap til en metodikk for å gjennomføre risikoanalyse.

\section{Digital kompetanse og beregningsorientert fysikk, behov i framtidas samfunn}
Engelsk tekst, skal oversettes til norsk (MHJ)
\subsection{Computing competence}

Computing means solving scientific problems using computers. It covers
numerical as well as symbolic computing. Computing is also about
developing an understanding of the scientific process by enhancing
algorithmic thinking when solving problems.  Computing competence has
always been a central part of the science and engineering
education.

Modern computing competence is about

\begin{itemize}
\item derivation, verification, and implementation of algorithms

\item understanding what can go wrong with algorithms

\item overview of important, known algorithms

\item understanding how algorithms are used to solve mathematical problems

\item reproducible science and ethics

\item algorithmic thinking for gaining deeper insights about scientific problems
\end{itemize}

\noindent
\subsection{Key elements in computing competence}

The power of the scientific method lies in identifying a given problem
as a special case of an abstract class of problems, identifying
general solution methods for this class of problems, and applying a
general method to the specific problem (applying means, in the case of
computing, calculations by pen and paper, symbolic computing, or
numerical computing by ready-made and/or self-written software). This
generic view on problems and methods is particularly important for
understanding how to apply available, generic software to solve a
particular problem.


Computing competence represents a central element
in scientific problem solving, from basic education and research to
essentially almost all advanced problems in modern
societies. Computing competence is simply central to further
progress. It enlarges the body of tools available to students and
scientists beyond classical tools and allows for a more generic
handling of problems. Focusing on algorithmic aspects results in
deeper insights about scientific problems.

Today's projects in science and industry tend to involve larger teams. Tools for reliable collaboration must therefore be mastered (e.g., version control systems, automated computer experiments f\
or reproducibility, software and method documentation).


\subsection{Overarching description}

Students of this program learn to use the computer as a laboratory for
solving problems in science and engineering.


A degree from this program gives the candidate a methodical training
in planning, conducting, and reporting large research projects, often
together with other students and university teachers.  THe projects
emphasize finding practical solutions, developing an intuitive
understanding of the science and the scientific methods needed to
solve complicated problems, use of many tools, and not least
developing own creativity and independent thinking. The thesis work is
a scientific project where the candidates learn to tackle a scientific
problem in a professional manner.  The program aims also at developing
a deep understanding of the role of computing in solving modern
scientific problems. A candidate from this program gains deep insights
in the fundamnetal role computations play in our advancement of
science and technology, as well as the role computations play in
society.

\subsection{Description of learning outcomes}

The power of the scientific method lies in identifying a given problem
as a special case of an abstract class of problems, identifying
general solution methods for this class of problems, and applying a
general method to the specific problem (applying means, in the case of
computing, calculations by pen and paper, symbolic computing, or
numerical computing by ready-made and/or self-written software). This
generic view on problems and methods is particularly important for
understanding how to apply available, generic software to solve a
particular problem.


Computing competence represents a central element
in scientific problem solving, from basic education and research to
essentially almost all advanced problems in modern
societies. Computing competence is simply central to further
progress. It enlarges the body of tools available to students and
scientists beyond classical tools and allows for a more generic
handling of problems. Focusing on algorithmic aspects results in
deeper insights about scientific problems.

The learning outcomes are subdivided in three general categories, knowledge, skills and general competence.

\begin{itemize}
\item \textbf{Knowledge}: A candidate from this program
\begin{itemize}

 \item has deep knowledge of the scientific method and computational science at an advanced level, meaning that the candidate
\begin{enumerate}

 \item has the ability to understand advanced scientific results in new fields

 \item has fundamental understanding of methods and tools

 \item can develop and apply advanced computational methods to scientific problems

 \item is capable of judging and analyzing all parts of the obtained scientific results

 \item can present results orally and in written form as scientific reports/articles

 \item can propose new hypotheses and suggest solution paths

 \item can generalize mathematical algorithms and apply them to new situations

 \item can link computational models to specific applications and/or experimental data

 \item can develop models and algorithms to describe experimental data

\item masters methods for reproducibility and how to link this to a sound ethical scienfitic conduct

\item has a thorough understanding of how computing is used to  solve  scientific problems
\item knows fundamental algorithms in computational science

\end{enumerate}

\noindent
 \item has a fundamental understanding and knowledge of scientific work, meaning that
\begin{enumerate}

 \item the candidate can develop hypotheses and suggest ways to test these

 \item can use relevant analytical, experimental and numerical tools and results to test the scientific hypotheses

 \item can generalize from numerical and experimental data to mathematical models and underlying principles

 \item can analyze the results and evaluate their relevance with respect to the actual problems and/or hypotheses

 \item can present the results according to good scientific practices

\end{enumerate}

\noindent
\end{itemize}

\noindent
\item \textbf{Skills}: A candidate from this program
\begin{itemize}

 \item has a deep understanding of what computing means, entailing several or all of the topics listed below
\begin{enumerate}

 \item knows the most fundamental algorithms involved, how to optimize these and perform statistical uncertainty quantification

 \item has overview of advanced algorithms and how they can be accessed in available software and how they are used to solve scientific problems

 \item has knowledge and understands high-performance computing elements: memory usage, vectorization and parallel algorithms

 \item can use effeciently high-performance computing resources, from compilers to hardware architectures

 \item understands approximation errors and what can go wrong with algorithms

 \item has knowledge of at least one computer algebra system and how it is applied to perform classical mathematics

 \item has extensive experience with programming in a high-level language (MATLAB, Python, R)

 \item has experience with programming in a compiled language (Fortran, C, C++)

 \item has experience with implementing and applying numerical algorithms in reusable software that acknowledges the generic nature of the mathematical algorithms

\item has experience with debugging software

\item has experience with test frameworks and procedures

\item has experience with different visualization techniques for different types of data

\item can critically evaluate results and errors

\item can develop algorithms and software for complicated scientific problems independently and in collaboration with other students

\item masters software carpentry: can design a maintainable program in a systematic way, use version control systems, and write scripts to automate manual work

\item understands how to increase the efficiency of numerical algorithms and pertinent software

\item has knowledge of stringent requirements to efficiency and precision of software
\item understands tools to make science reproducible and has a sound ethical approach to scientific problems

\end{enumerate}

\noindent
\end{itemize}

\noindent
\item \textbf{General competence}: A candidate from this program
\begin{itemize}

 \item is able to develop professional competence through the thesis work, entailing:
\begin{enumerate}

 \item mature professionally and be able to work independently

 \item can communicate in a professional way scientific results, orally and in written form

 \item can plan and complete a research project

 \item can develop a scientific intuition and understanding that makes it possible to present and discuss scientific problems, results and uncertainties

\end{enumerate}

\noindent
 \item is able to develop virtues, values and attitudes that lead to  a better understanding of ethical aspects of the scientific method, as well as promoting central aspects of the scientific method to society. This means for example that the candidate
\begin{enumerate}

 \item can reflect on and develop strategies for making science reproducible and to promote the need for a proper ethical conduct

 \item has a deep understanding of the role basic and applied  research and computing play for progress in society

 \item is able to promote, use and develop version control tools in order to make science reproducible

 \item is able to critically evaluate the consequences of own research and how this impacts society

 \item matures an understanding of the links between basic and applied research and how these shape, in a fundamental way,  progress in science and technology

 \item can develop an understanding of the role research and science can play together with industry and society in general

 \item can reflect over and develop learning strategies for life-long learning.
\end{enumerate}

\noindent
\end{itemize}

\noindent
\end{itemize}

\noindent
The candidate will have developed a critical understanding of the scientific methods which have been studied, has a better understanding of the scientific process per se as well as having developed perspectives for future work and how to verify and validate scientific results.



\section{Eksperimentell kompetanse, behov i framtidas samfunn}

\subsection{Kunnskap om eksperimentdesign}
Et eksperiment bør være designet slik at det er egnet til å teste en gitt hypotese.
Det må være en protokoll eller plan for eksperimentet.
Et eksperiment må isolere den eller de egenskapene som vil gjøre det mulig å bekrefte eller avkrefte hypotesen.
Det kan være vanskelig å designe et eksperiment som tester en gitt hypotese.

Det er nyttig for arbeidsgivere at arbeidstakerne har kunnskap om framgangsmåter som kan brukes for å designe et eksperiment, og at de har noe trening i å designe et eksperiment som gir den nøyaktigheten som er nødvendig for å kunne teste hypotesen. I en jobbsituasjon vil det vanligvis være begrensete ressurser som kan brukes på et eksperiment, f.eks. tid og/eller penger.
Det er nyttig for arbeidsgivere at en arbeidstaker har opplæring i følgende ferdigheter:
\begin{itemize}
  \item Bestemme hvilken nøyaktighet et eksperimentet må ha for å kunne avkrefte eller bekrefte en hypotese med en bestemt grad av sikkerhet.
  \item Diskutere alternative måleteknikker, identifisere hvilke(n) som er egnet, og finne egnet måleutstyr og oppsett.
  \item Vurdere antall målinger/replikater som må finnes for å for å ha en akseptabel verdi for tilfeldig feil, beregne feilestimat for de målte verdiene, og vurdere om noen dominerende feilkilder kan/bør minkes.
  \item Sørge for at eksperimentoppsettet vil gi gyldige data: at måleutstyr er kalibrert og at oppsettet tilfredsstiller en ``system suitability test'' før det brukes til å samle inn data.
  \item Bruke automatiserte løsninger for datainnsamling og evt. dataanalyse.
  \item Avpasse eksperimentet etter begrensninger på tid og kostnad og ønsket nøyaktighet på resultatet.
\end{itemize}

\subsection{Kunnskap om vanlig brukte måleteknologier}
Innenfor eksperimentelle ferdigheter vil arbeidsgivere ha nytte av at arbeidstakerne er kjent med de aller vanligste eksperimentelle teknikkene som brukes innenfor fagområdet.
I tillegg vil arbeidsgivere ha nytte av at arbeidstakerne har kunnskaper som setter dem i stand til å sette opp automatisk datainnsamling fra et måleoppsett.
Arbeidsgivere forventer ikke at nylig uteksaminerte studenter har detaljkunnskap om den nyeste teknologien og metodikken innenfor et smalt fagfelt eller bruksområde.
I næringsliv og anvendt forskning forventes det at alle ansatte har grunnkunnskaper som setter dem i stand til å forstå og anvende ny teknologi og metodikk ettersom den blir tilgjengelig og er til nytte i arbeidsoppgavene som skal utføres.
Innen akademia forventes det at ansatte i tillegg selv er i stand til å utvikle ny teori, teknologi og/eller metodikk.

\section{Dagens studieprogram}
\label{Sammenlikning}

I denne delseksjonen trekker vi frem noen av del elementene fra dagens studieprogram som arbeidsgruppen har funnet spesielt interessant å kommentere.
Noen av disse elementene er generelle, mens andre er trukket frem spesielt for eksperimentell fysikk eller beregningsorientert fysikk.

{\color{red} Mening om form: kan vi bruke 3dje person flertall?}

\subsection{Sammenfatte resultater av en undersøkelse i en rapport}
\label{Rapport}
{\color{red} KarlYngve: Dette er også relevant for beregningsorientert fysikk, og man kunne derfor generalisert delseksjonen.}

Undervisningen i eksperimentelle ferdigheter i MTFYMA legger liten vekt på trening i å skrive vitenskapelige rapporter. Det skal skrives rapporter etter mange av laboratoriearbeidene, men langt de fleste av disse rapportene ser ut til å bestå av å svare på forhåndsstilte spørsmål.

I faget TFY4260 skal studentene skrive en vitenskapelig rapport og får artikkelen av Ecarnot et al.~som læremateriale. Artikkelen er en dekkende og tydelig innføring i prinsippene og formelle krav til en vitenskapelig artikkel. Innholdet er svært relevant for vitenskapelige rapporter. I faget TBT4102 gir kompendiet ``Laboratoriekurs i TBT4102 Biokjemi 1'' en innføring i strukturen til en vitenskapelig rapport, og en forklaring på hvilken informasjon som hører hjemme i de forskjellige kapitlene i en vitenskapelig rapport. Studentene skal skrive to rapporter i dette faget. Dette betyr at studenter som følger studieprogrammet i Biofysikk og medisinsk teknologi får god trening i å skrive vitenskapelige rapporter.

Kapittel 9 i heftet ``TFY 4190 Instrumentering'' i faget TFY4190 gir noen retningslinjer for en vitenskapelig rapport. Skrivet ``Requirements \& guidelines for reports in Solid State Physics'' av Dag W.~Breiby (2020) brukes i faget TFY4220 og beskriver noen retningslinjer og noen vanlige feil i vitenskapelige rapporter. Dette betyr at studentene som følger studieprogrammet Teknisk fysikk får mindre trening i, og undervisningsmateriell for, å skrive vitenskapelige rapporter enn studentene som følger studieprogrammet for biofysikk.

Det er svært nyttig i arbeidslivet å kunne skrive en mer eller mindre vitenskapelig rapport.
Det bør vurderes om studentene kan få mer undervisning og praktisk øvelse i rapportskriving. Det kan vurderes å gi øvelse i rapportskriving allerede fra første studieår og jevnlig gjennom studieløpet.

Øvelse i å skrive vitenskapelige rapporter krever at det eksperimentelle arbeidet egner seg for rapportskriving. Mange av eksperimentene som studentene gjør har preg av å være demonstrasjoner av fysiske effekter. Dette er både gøy og godt egnet til å hjelpe studentene å forstå teori. Likevel, slike gode demonstrasjoner er ikke alltid godt egnet for å trene på å skrive rapporter. En vitenskapelig rapport er skrevet for å besvare et klart formulert forskningsspørsmål eller hypotese. Det er svært få, om noen, av de oppgavene studentene får som formulerer et klart forskningsspørsmål eller hypotese. Mangel på et klart forskningsspørsmål kan gjøre det krevende å skrive en rapport som følger de standardiserte retningslinjene. Det bør vurderes om noen utvalgte laboratorieoppgaver skal omformuleres eller omarbeides slik at de blir godt egnet som grunnlag for en vitenskapelig rapport.

Det ser ikke ut til at det brukes noe felles undervisningsmateriell som forklarer studenter på alle studieprogrammene de grunnleggende reglene for en vitenskapelig rapport. Artikkelen som brukes i faget TFY4260 kan fungere som en slik felles materiell. Alternativt materiell kan være tilsvarende, men litt kortere, artikler \cite{Lapin1994} og \cite{Senturia2003}.

\subsection{Erfaring med å jobbe både i prosjektform og selvstendig}
Undervisningen i eksperimentelle ferdigheter foregår i små grupper i fagene FY1001, FY1003, TFY4163, TFY4165, TFY4220, TFY4260, og TMT4110. I fagene TFY4185, TFY4190, og TBT4102 gjennomføres et større arbeid som et prosjekt som går over 4--5 uker. I tillegg har studentene ved MTFYMA faget Eksperter i Team (EITXXXX?). Disse fagene gir relevant trening i å jobbe i prosjektform sammen med andre. Denne erfaringen er nyttig i arbeidslivet.
Det er uklart om studenter som ikke følger noen av fagene TFY4185, TFY4190, eller TBT4102 får tilsvarende trening i å jobbe i prosjektform sammen med andre over flere uker.

{\color{red} KarlYngve: Her er vel eksperter i team viktig å nevne og virker å være veldig relevant.}


{\color{red} KarlYngve: Jeg er usikker på dette, men kan det være en lignende, men motsatt problemstilling for beregningsorientert fysikk? Altså at studenter som går numerisk rettet har litt lite gruppearbeid?}

\subsection{Formulere et forskningsspørsmål eller en testbar hypotese}
De aller fleste av de eksperimentelle oppgavene i fagene TMT4110, FY1001, FY1003, TFY4163, TFY4165, TFY4185, TFY4220 og TFY4190 har formulerte læringsmål. Imidlertid er det få eller ingen av oppgavene i disse fagene som presenterer et klart formulert forskningsspørsmål eller testbar hypotese som grunnlag for eksperimentet. Det ser heller ikke ut til at noen eksperimentelle oppgaver i programmet MTFYMA trener studentene i selv å formulere en testbar hypotese.
Mange av de eksperimentelle oppgavene har som læringsmål å ``studere'' eller ``undersøke'' en egenskap. Disse formuleringene er ikke testbare: det er ikke mulig å vite om målet er oppfylt eller ikke. Dette gjør at målet med selve eksperimentet er implisitt. Et implisitt mål er uklart, og gjør det senere vanskelig å skrive en god vitenskapelig rapport (se \cref{Rapport}).

\subsection{Kunnskap om god dokumentasjonspraksis}
Undervisningen i eksperimentelle ferdigheter bruker en generell ressurs i Perssons kompendium \cite{Persson2020}, men også mange tilleggsskriv fra de forskjellige faglærene. Eksempler på tilleggs-skriv er heftet til Persson i emnet FY1001 \cite{Persson2020FY1001}. Ingen av kompendiene eller skrivene ser ut til å gi noen tydelige henvisninger til etablerte retningslinjer GDP eller GLP. Praktiske henvisninger til GDP kan være f.eks. Wikipedia \cite{WikiGDP} eller kapittelet ``Good Documentation Practices'' i \cite{Davani2017}. En praktisk henvisning til GLP kunne være f.eks. deler av OECD sine prinsipper \cite{OECD1997}.

Det generelle kompendiet \cite{Persson2020} beskriver en bruk av journal som ikke er vanlig i næringsliv og oppdragsforskning. Prinsippene for dokumentasjon av ideer, protokoller, planer, eksperimentelle oppsett, beregninger og resultatvurderinger anvendes i næringsliv og oppdragsforskning. Imidlertid er det svært vanlig i arbeidslivet at selve dokumentasjonen må være tilgjengelig for flere personer, for eksempel prosjektmedarbeidere. Dette sikrer at prosjektets fremdrift er robust og ikke avhengig av at alle er på jobb hele tiden. Derfor er det vanlig å lagre det meste av dokumentasjon i en elektronisk ressurs og ikke i en håndskrevet journal. En elektronisk ressurs kan for eksempel være et elektronisk dokumenthåndteringssystem eller en filserver.
Prinsippet om samtidig datainnsamling står fast i eksperimentelt arbeid. Automatisert datainnsamling er svært vanlig, og kan for eksempel sørge for at data samles på en felles filserver der de er beskyttet mot modifikasjon.

Manuell datainnsamling gjøres også, ofte i håndskrevne eller elektroniske journaler. Det er vanlig i en håndskrevet journal å henvise til dokumentasjonen for protokoller, beregninger, resultatvurderinger og liknende. Kompendiet i TMT4110 gir gode forslag til hvordan studentene skal strukturere tabeller og journal for samtidig innsamling av måledata underveis i eksperimentet. Dermed gir labøvelsene i TMT4110 studentene mye trening i manuell datainnsamling. Faget gir studentene en grunnopplæring i manuell datainnsamling som er relevant og tilstrekkelig for arbeidsgivere. Studieprogrammet har ikke en liknende, veiledet trening i manuell datainnsamling i noen av fysikkfagene.

\subsection{Kjennskap til risikoanalyse}
Kompendiet i faget TMT4110 gir en innføring i FMEA-metodikk for risikoanalyse og viser risikomatriser. Kompendiet gir en innføring i merking av kjemikalier og bruk av material safety data sheets (MSDS). HMS er behandlet som eget punkt i hver labøvelse. Labøvelsene 5--10 i faget TFY4190 gir trening i å gjennomføre en risikoanalyse etter FMEA-metodikk. HMS blir nevnt i røntengdiffraksjonsoppgaven i TFY4220, der studentene skal måle røntgenstrålingen fra instrumentet.
Til sammen gir disse fagene studentene en grunnopplæring i risikoanalyse som er relevant og tilstrekkelig for arbeidsgivere.

\subsection{Kunnskap om vanlig brukte måleteknologier}
% Gis det tilstrekkelig undervisning i tolkning av storedatasett og/eller artificial intelligence i numerikkfagene? (Kari)
Faget FY1001 gir studentene kunnskap om å logge data fra en sensor og bruke sensorprodusentens egne software. Fagene TFY4185 og TFY4190 utvikler disse kunnskapene betydelig og gir kunnskap om bruk av Labview og custom programvare for sensorer, sensorer, og datalogging. Disse kunnskapene er svært nyttige og direkte anvendbare i mange typer jobber innenfor forskjellige fagfelt.

Faget TFY4220 gir studentene kunnskaper om FTIR-spektrometri, røntgendiffraksjon og TEM. Alle disse teknikkene er vanlige å bruke innenfor forskning og utvikling og i enkelte typer industri. SEM er også mye brukt i flere bransjer i arbeidslivet. Det kan vurderes om studentene som følger studieprogrammet i Teknisk fysikk skulle ha en introduksjon til SEM på samme måte som blir gitt i faget TFY4260.
Faget TMT4110 gir studentene kunnskaper om et bredt spekter av vanlige teknikker i kjemisk laboratoriearbeid. Teknikkene er vanlige å bruke innenfor enkelte typer industri.

Faget TFY4260 gir studentene kunnskaper om cellekultivering, flowcytometri, måling av bioimpedans, farging, merking, og forskjellige mikroskopiteknikker. Mikroskopiteknikker er svært vanlige å bruke i mange typer jobber innenfor forskjellige fagfelt. Farging, merking, cellekultivering og flowcytometri er svært vanlige teknikker innenfor forskning og utvikling i biologi, medisin og biofysikk.

Faget TBT4102 gir studentene kunnskaper om kromatografi og elektroforese. Begge disse teknikkene er vanlige å bruke innenfor forskning og utvikling og i enkelte typer industri.

To av laboratorieoppgavene i faget FY1001 og alle laboratorieoppgavene i fagene FY1003, TFY4163 og TFY4165 har preg av å være demonstrasjoner som hjelper studentene å forstå teorien de leser. Disse laboratorieoppgavene ser ut til å støtte teoriundervisningen og ikke gi studentene faglige kunnskaper utover det som også formidles gjennom forelesningene.

\subsection{Kunnskap om eksperimentdesign}
Med to unntak er undervisningen i eksperimentelle ferdigheter organisert som en trinnvis øvelse som studentene ledes gjennom. Selv om studentene selv skal gjennomføre det praktiske arbeidet, legger undervisningen ikke til rette for at studentene selv skal finne et egnet måleoppsett eller skrive protokoll. Unntakene er fagene TFY4190 og TBT4102. I TFY4190 skal studentene planlegge, og gjennomføre et eksperiment og skrive rapport, og i TBT4102 skal studentene skrive protokoll, gjennomføre eksperimentelt arbeid, og skrive rapport.

Oppgaver med trinnvise instruksjoner kan egne seg godt for å skape forståelse av en fysisk egenskap eller mekanisme. Likevel, i arbeidslivet er det sannsynlig at mange studenter forventes å kunne omsette komplekse oppgaver eller problemstillinger til et eksperiment som kan hjelpe til å ta en beslutning, komme videre med oppgaven eller kaste lys over problemstillingen.

Det er omfattende å designe et eksperiment fordi det må gjøres mange vurderinger før protokollen er endelig. Likevel, det kunne tenkes at en del momenter ved design av et eksperiment kunne bli en del av allerede eksisterende laboratorieoppgaver. Dette ville presentere studentene for hvilke vurderinger som kan være viktige når et eksperiment skal designes, og gi noe trening i strategier for å designe et eksperiment.

Vurderinger som kunne være aktuelle å legge inn som en del av en eller flere laboratorieoppgaver:
\begin{itemize}
  \item På forhånd bestemme hvilken nøyaktighet måleresultatet må ha for at resultatet skal ha verdi. Eksempel: hvor nøyaktig må eksperimentet ``bestemme vekt på koffert'' være for at en er sikker på å ikke måtte betale for overvektig bagasje på flyet.

  \item Vurdere alternativer for eksperimentelle metoder: hvilke kan og hvilke kan ikke gi den nødvendige nøyaktigheten i målingene. Eksempler: må dimensjoner måles med mikrometerskrue eller linjal? Hvor mange desimaler må en vektmåling ha? Hvilken spesifikasjon må en trykkmåler ha for å kunne bestemme høyden på Nidarosdomen med usikkerhet $\pm\SI{1}{\meter}$? Feilanalyse er tema flere ganger i MTFYMA: i \cite{Persson2020}, i Cavendish’ eksperiment i FY1001, i oppgaven om overflatespenning i TFY4163, i oppgaven om Einsteintemperatur i TFY4165, og i oppgaven om røntgendiffraksjon i TFY4220. I laboratorieoppgavene gjøres imidlertid feilanalysen etter at eksperimentet er gjort. I arbeidslivet vil det være naturlig å vurdere nøyaktighet i eksperimentet på forhånd for å kunne avgjøre om den eksperimentelle metoden kan brukes eller ikke.

  \item Sammenlikne ressursbruk ved alternative eksperimentelle metoder: hvor mye tid og penger trengs for å gjennomføre eksperimentet?

  \item Bestemme det nødvendige antall paralleller/replikater som må gjøres av et eksperiment for å få størrelsen på tilfeldig feil til å bli ``lav nok'' og bestemme hva som er ``lavt nok''.

  \item Finne om det er én verdi i et beregnet uttrykk som dominerer usikkerheten i måleresultatet. Vurdere om det er et annet måleutstyr som bør skiftes ut for å bedre nøyaktigheten. Eksempel: en mer nøyaktig vekt, en sensor med høyere oppløsning.

  \item Se effekten av å bruke kalibrerte måleinstrumenter og mulig konsekvens av ikke-kalibrerte måleinstrumenter. Ingen av laboratorieoppgavene gir en demonstrasjon av mulig konsekvens av å bruke ikke-kalibrerte måleinstrumenter, selv om flere oppgaver nevner at måleutstyret er kalibrert.

  \item Se effekten av å gjennomføre en ``system suitability test'', f.eks. å se at spekteret til en kjent prøve er innenfor fastsatte krav før utstyret brukes til å måle en ukjent prøve, eller måle en kjent vektstandard før en ukjent prøve.  En ``system suitability test'' gjøres med silisiumpulver i røntgendiffraksjonsoppgaven i TFY4220. Oppgaven om Einsteintemperaturen til aluminium i TFY4165 nevner behovet for vatring og tarering av vekt. Det kan vurderes om konseptet ``System suitability test'' bør introduseres i et tidligere årskurs slik at alle studentene på MTFYMA blir kjent med det.
\end{itemize}



\section{Integrasjon av beregninger, teori og eksperiment i studieløpet}

\section{Bærekraft}

\section{Etiske betraktninger}

\section{Arbeidslivets behov}

\section{Sammenlikning med undervisningsopplegg ved andre læresteder}

\section{Studentenes opplevelse av nåværende undervisningsopplegg}

\section{Anbefalinger}
Basert på arbeidsgivernes behov (\cref{Behov}) og inneholdet i nåværende undervisningsopplegg (\cref{Sammenlikning}), anbefaler den eksterne arbeidsgruppen at ledelsen for studieprogrammene ser nærmere på noen aspekter ved undervisningen i eksperimentelle ferdigheter.
\begin{itemize}
  \item Vurdere å øke antallet vitenskapelige rapporter som studentene skal skrive, og gi studentene en ekstern ressurs, for eksempel en bok eller artikkel, om gjeldende retningslinjer for skriving av vitenskapelige rapporter.
  \item Vurdere å gjennomføre noen laboratorieoppgaver og/eller vitenskapelige rapporter som individuelt arbeid.
  \item Vurdere å skille tydelig mellom laboratorieoppgaver som skal tjene som demonstrasjon av en fysisk effekt og laboratorieoppgaver som er eksperimenter som skal svare på et forskningsspørsmål eller hypotese. For de siste, vurdere å eksplisitt formulere forskningsspørsmålet/hypotesen.
  \item Vurdere å gi studentene kjennskap til eksterne ressurser om oppdaterte retningslinjer for god dokumentasjonspraksis (GDP) og god laboratoriepraksis GLP).
  \item Vurdere å gi studentene en beskrivelse av gjeldende praksis i journalbruk og journalføring, slik journaler er vanlig brukt i arbeidslivet i dag.
  \item Vurdere å ta inn elementer av eksperimentdesign og protokollskriving i laboratorieoppgavene. Slike elementer kan være å vurdere ulike alternative eksperimentoppsett med tanke på nøyaktighet, tidsbruk og kostnader, forhåndsberegning av måleusikkerhet og å fastsette antall replikater som er nødvendig, og å vurdere behovet for kalibrering av instrumenter og bruk av ``system suitability tests'' i et eksperimentelt oppsett.
\end{itemize}

\appendix
\section{Mandat for ekstern arbeidsgruppe for evaluering av studieprogrammene MTFYMA og BFY}
\subsection{Bakgrunn}
I henhold til NTNUs system for kvalitetssikring skal alle studieprogram evalueres hvert femte år. Arbeidsgruppen er en del av en slik periodisk evaluering. Fra høsten 2019 innførte Institutt for fysikk såkalte ferdighetsstrenger innen 1) Numerisk fysikk og programmering (NP) og 2) Eksperimentell fysikk (XF). Hensikten er å få til en mer planmessig utvikling av studentenes ferdigheter innenfor disse områdene. 
\subsection{Mål}
Studieprogrammene ønsker å kartlegge eksisterende aktivitet innenfor disse områdene, samt få en ekstern evaluering, for å bruke dette som grunnlag for videreutvikling av ferdighetsstrengene.
Vi ønsker spesielt å få vurdert
\begin{itemize}
  \item I hvilken grad imøtekommer utdanningene samfunnets behov innenfor beregningsorientering og eksperimentell fysikk.
  \item Hvordan sammenlikner den eksisterende og den planlagte utdanningen med hvordan dette adresseres på andre læresteder (både i Norge og internasjonalt).
  \item Er det noe spesielt som mangler innenfor disse områdene i et bærekraftperspektiv?
\end{itemize}
\subsection{Organisasjon}
Den eksterne arbeidsgruppa består av minst 6 representanter, 3 fra hvert av fagområde (beregningsorientering og eksperimentell fysikk). Det skal være minst en fra akademia og en fra industri innen hvert fagområde.
Ett av medlemmene i gruppa utpekes som leder og er ansvarlig for gjennomføring i henhold til mandatet.
\subsection{Gjennomføring/fremdrift/rapportering}
Studieprogrammene gjennomfører først en intern kartlegging av nåværende innhold i emnene relatert til ferdighetsstrengene. Studieprogramlederne systematiserer data og skriver et kort sammendrag. Denne kartleggingen ble ferdigstilt primo september 2020.
Den eksterne arbeidsgruppa baserer sin evaluering basert på den interne kartleggingen. En studentgruppe vil arbeid parallelt med den eksterne komiteen med samme tema. Om ønskelig kan den eksterne arbeidsgruppen arrangere et møte med studentgruppen for å få studentenes perspektiv på innholdet av beregningsorientering og eksperimentell fysikk i utdanningene. Det anbefales å gjøre dette møtet digitalt.
Hvert medlem i arbeidsgruppa skriver en egen vurdering av innholdet av ferdighetsstrengene innenfor sitt tema. Denne vurdering er basert på dokumentasjon fra den interne kartleggingen og skal besvare spørsmålene stilt over. Den interne kartleggingen inneholder også en del konkrete spørsmål som ønskes besvart (se kartleggingsdokument).
I tillegg skal leder for arbeidsgruppa utarbeide ett sammendrag av arbeidsgruppas vurderinger, inkludert studentenes perspektiv på utdanningen. 
Frist for endelig rapport er 15. desember 2020.
Arbeidsgruppa forventes å ha minst ett møte før sammendrag og endelig rapport sammenstilles.
Arbeidsgruppa forventes å delta på et seminar i Trondheim for å legge frem arbeidsgruppas rapport og diskutere funn. Dette vil gjennomføres i løpet av januar 2021.
\subsection{Ressursbruk}
Medlemmer i arbeidsgruppa honoreres etter gjeldende satser.
Utgifter til angitte reiser og opphold dekkes.


\bibliographystyle{unsrt}
\bibliography{References}

\end{document}
