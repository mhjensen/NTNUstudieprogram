\documentclass{article}
\usepackage[T1]{fontenc}
\usepackage[utf8]{inputenc}
\usepackage[norsk]{babel}
\usepackage[a4paper, margin=3cm]{geometry}
\usepackage{siunitx}
\usepackage{hyperref}
\usepackage{cleveref}
\usepackage{xcolor}

\title{En ekstern arbeidsgruppes evaluering av NTNUs studieprogram MTFYMA og BFY}
\author{Arbeidsgruppa består av: \\ Dag Kristian Dysthe (UiO), \\ Sverre Gullikstad Johnsen (SINTEF), \\ Morten Hjorth-Jensen (UiO), \\ Karl Yngve Lervåg (SINTEF), \\ Kari Schjølberg-Henriksen (GE Healthcare), \\ Ingve Simonsen (NTNU), \\ Paal Skjetne (SINTEF)}
\date{\today}

\newcommand{\crefrangeconjunction}{--}
\crefname{section}{seksjon}{seksjoner}
\crefname{figure}{figur}{figurer}

\begin{document}

\maketitle
\tableofcontents

\begin{abstract}
\end{abstract}

\section{Introduksjon {\color{red} Morten+Alle}}
Eksperiment, teori og beregninger er de sentrale bærebjelkene i dagens fysikkfag. Samspillet mellom  eksperiment, beregninger og teori har lagt grunnlaget for flere av våre innsikter om  viktige naturvitenskapelige prosesser  samt vår grunnlegende forståelse av sentrale naturlover og utvikling av vårt moderne teknologiske samfunn.
En grunnleggende forståelse av eksperiment, teori og beregninger er essensielle element i teknologisk innovasjon og utvikling, og legger grunnlaget for vårt framtidige kunnskapsbaserte samfunn. 

Numeriske modeller og beregninger spiller en sentral rolle i dagens naturvitenskapelige og teknologiske utvikling. Kompliserte eksperimenter erstattes mer og mer av numeriske beregninger. Kunnskap om og innsikt og evne til å behandle og analysere store datasett spiller også en sentral rolle. Mange viktige oppdagelser og teknologiske nyvinninger ville ikke vært mulig uten omfattende numerisk modellering og dataanalyse.  

Vitenskapelige beregninger har som mål å utvikle nøyaktige metoder og modeller som setter oss i stand til å studere komplekse naturvitenskapelige og teknologiske system og fenomen. Mange av disse er såpass komplekse at fysiske eksperiment fort blir for kostbare eller nesten umulige å gjennomføre i et laboratorium. Numeriske beregninger og modellering spiller dermed en sentral rolle i moderne vitenskap og teknologisk utvikling. 
Digital kompetanse, innsikter og ferdigheter bør dermed være naturlige element i et naturvitenskapelig studieprogram. For et eksperimentelt fag som fysikk, åpner det seg også mange nye pedagogiske muligheter i integrasjonen av numeriske beregninger, dataanalyse og eksperiment. Mange eksperiment kan nå gjennomføres ved hjelp av f.eks.\ enkle applikasjoner i smarttelefoner~\cite{phyphox} og enkle apparat som Arduino~\cite{arduino}. Dette åpner for en tettere integrasjon av beregninger, teori, dataanalyse og eksperiment i utdanningsløpet. På sikt muliggjør dette en dypere forståelse av fysikkfagets egenart og den vitenskapelige metode på et tidligere stadium i studieløpet.

Denne arbeidsgruppa er bedt spesielt om å se på følgende punkter:
\begin{itemize}
  \item I hvilken grad imøtekommer utdanningene samfunnets behov innenfor beregningsorientert og eksperimentell fysikk?
  \item {\color{green!50!black}Hvordan er den eksisterende og planlagte utdanningen sammenlignet med tilsvarende program på andre læresteder (både i Norge og internasjonalt)? (KYL: Jeg foreslår denne formuleringen i stedet for den under.)}
  \item {\color{red}Hvordan sammenlikner den eksisterende og den planlagte utdanningen med hvordan dette adresseres på andre læresteder (både i Norge og internasjonalt).}
  \item Er det noe spesielt som mangler innenfor disse områdene i et bærekraftperspektiv?
  
  {\color{red} KarlYngve: Hva legges i ``disse områdene'', og hva er man egentlig ute etter med dette spørsmålet?}

  {\color{yellow!50!black}MHJ: Passe på dette i rapporten, link med planlagt skisse.}
\end{itemize}

\subsection{Rapportens struktur}
Rapporten inneholder en beskrivelse av ferdigheter, sentrale kompetanser og kunnskaper som arbeidsgruppa ser som sentrale i nåværende og framtidige naturvitenskapelige og teknologiske arbeidsmiljø. Beskrivelsen av disse ulike sentrale ferdighetene, kompetansene og kunnskapene  blir deretter satt i sammenheng med  dagens studieprogram og situasjonen ved andre utvalgte universitet. Inntrykk fra hvordan dagens studenter oppfatter det nåværende studieløpet er også inkludert. 

Basert på dette presenterer vi ulike anbefalinger om hvordan en kan integrere teori, beregninger og eksperiment i et studieløp i fysikk. Vårt overordna  mål er spille inn anbefalinger om hvordan en kan  imøtekomme framtidige behov i et høgteknologisk arbeidsliv, enten det dreier seg om  offentlig eller privat sektor. 

\subsection{Om arbeidsgruppa}
Vurderingene og anbefalingene i denne rapporten er basert på den eksterne arbeidsgruppas kunnskap om behov i et moderne og høgteknologisk arbeidsliv. Våre erfaringer spenner fra akademia til offentlig og privat sektor. 
Medlemmene i den eksterne arbeidsgruppen har omfattende erfaring fra både undervisning, forskning og teknologisk utvikling i offentlig og privat sektor. Alle 
har vært med å veilede nyansatte og studenter på både bachelor-, master- og PhD-nivå.

%Medlemmenes sektortilknytning over tid:

% Jeg har lagt inn min egen arbeidstid i forskjellige sektorer. Jeg foreslår å ta med dette for at studieledelsen kan se på våre anbefalinger samtidig som de tar hensyn til hvor studentene ofte begynner å jobbe. Hvis dere er enige, kan dere legge til deres egne år og sektorer. Det blir sikkert mange hundre år til sammen (Kari)

%\begin{itemize}
%  \item Akademia (9 år)
%  \item Oppdragsforskning med nær kontakt med industri (25 år)
%  \item Industri (2 år)
%\end{itemize}

%{\color{red} KarlYngve: Dette må forklares litt nærmere om vi har det med; det er ikke åpenbart at årene i parentes her er hvor mange år medlemmene totalt har vært involvert i de nevnte sektorene. Paal: enig, jeg kan legge beslag på 22.5 år av de 25 under oppdragsforskning ... :-0 }

\section{Dagens studieprogram {\color{red} Karl Yngve+alle}}
\label{Sammenlikning}

I denne delseksjonen trekker vi frem noen av del elementene fra dagens studieprogram som arbeidsgruppen har funnet spesielt interessant å kommentere.
Noen av disse elementene er generelle, mens andre er trukket frem spesielt for eksperimentell fysikk eller beregningsorientert fysikk.

\subsection{Sammenfatte resultater av en undersøkelse i en rapport}
\label{Rapport}
{\color{red} KarlYngve: Dette er også relevant for beregningsorientert fysikk, og man kunne derfor generalisert delseksjonen.}

Undervisningen i eksperimentelle ferdigheter i MTFYMA legger liten vekt på trening i å skrive vitenskapelige rapporter. Det skal skrives rapporter etter mange av laboratoriearbeidene, men langt de fleste av disse rapportene ser ut til å bestå av å svare på forhåndsstilte spørsmål.

I faget TFY4260 skal studentene skrive en vitenskapelig rapport og får artikkelen av Ecarnot et al.~som læremateriale. Artikkelen er en dekkende og tydelig innføring i prinsippene og formelle krav til en vitenskapelig artikkel. Innholdet er svært relevant for vitenskapelige rapporter. I faget TBT4102 gir kompendiet ``Laboratoriekurs i TBT4102 Biokjemi 1'' en innføring i strukturen til en vitenskapelig rapport, og en forklaring på hvilken informasjon som hører hjemme i de forskjellige kapitlene i en vitenskapelig rapport. Studentene skal skrive to rapporter i dette faget. Dette betyr at studenter som følger studieprogrammet i Biofysikk og medisinsk teknologi får god trening i å skrive vitenskapelige rapporter.

Kapittel 9 i heftet ``TFY 4190 Instrumentering'' i faget TFY4190 gir noen retningslinjer for en vitenskapelig rapport. Skrivet ``Requirements \& guidelines for reports in Solid State Physics'' av Dag W.~Breiby (2020) brukes i faget TFY4220 og beskriver noen retningslinjer og noen vanlige feil i vitenskapelige rapporter. Dette betyr at studentene som følger studieprogrammet Teknisk fysikk får mindre trening i, og undervisningsmateriell for, å skrive vitenskapelige rapporter enn studentene som følger studieprogrammet for biofysikk.

Det er svært nyttig i arbeidslivet å kunne skrive en mer eller mindre vitenskapelig rapport.
Det bør vurderes om studentene kan få mer undervisning og praktisk øvelse i rapportskriving. Det kan vurderes å gi øvelse i rapportskriving allerede fra første studieår og jevnlig gjennom studieløpet.

Øvelse i å skrive vitenskapelige rapporter krever at det eksperimentelle arbeidet egner seg for rapportskriving. Mange av eksperimentene som studentene gjør har preg av å være demonstrasjoner av fysiske effekter. Dette er både gøy og godt egnet til å hjelpe studentene å forstå teori. Likevel, slike gode demonstrasjoner er ikke alltid godt egnet for å trene på å skrive rapporter. En vitenskapelig rapport er skrevet for å besvare et klart formulert forskningsspørsmål eller hypotese. Det er svært få, om noen, av de oppgavene studentene får som formulerer et klart forskningsspørsmål eller hypotese. Mangel på et klart forskningsspørsmål kan gjøre det krevende å skrive en rapport som følger de standardiserte retningslinjene. Det bør vurderes om noen utvalgte laboratorieoppgaver skal omformuleres eller omarbeides slik at de blir godt egnet som grunnlag for en vitenskapelig rapport.

Det ser ikke ut til at det brukes noe felles undervisningsmateriell som forklarer studenter på alle studieprogrammene de grunnleggende reglene for en vitenskapelig rapport. Artikkelen som brukes i faget TFY4260 kan fungere som en slik felles materiell. Alternativt materiell kan være tilsvarende, men litt kortere, artikler \cite{Lapin1994} og \cite{Senturia2003}.

\subsection{Erfaring med å jobbe både i prosjektform og selvstendig}
Undervisningen i eksperimentelle ferdigheter foregår i små grupper i fagene FY1001, FY1003, TFY4163, TFY4165, TFY4220, TFY4260, og TMT4110. I fagene TFY4185, TFY4190, og TBT4102 gjennomføres et større arbeid som et prosjekt som går over 4--5 uker. I tillegg har studentene ved MTFYMA faget Eksperter i Team (EITXXXX?). Disse fagene gir relevant trening i å jobbe i prosjektform sammen med andre. Denne erfaringen er nyttig i arbeidslivet.
Det er uklart om studenter som ikke følger noen av fagene TFY4185, TFY4190, eller TBT4102 får tilsvarende trening i å jobbe i prosjektform sammen med andre over flere uker.

{\color{red} KarlYngve: Her er vel eksperter i team viktig å nevne og virker å være veldig relevant.}

{\color{red} KarlYngve: Jeg er usikker på dette, men kan det være en lignende, men motsatt problemstilling for beregningsorientert fysikk? Altså at studenter som går numerisk rettet har litt lite gruppearbeid?}

\subsection{Formulere et forskningsspørsmål eller en testbar hypotese}
De aller fleste av de eksperimentelle oppgavene i fagene TMT4110, FY1001, FY1003, TFY4163, TFY4165, TFY4185, TFY4220 og TFY4190 har formulerte læringsmål. Imidlertid er det få eller ingen av oppgavene i disse fagene som presenterer et klart formulert forskningsspørsmål eller testbar hypotese som grunnlag for eksperimentet. Det ser heller ikke ut til at noen eksperimentelle oppgaver i programmet MTFYMA trener studentene i selv å formulere en testbar hypotese.
Mange av de eksperimentelle oppgavene har som læringsmål å ``studere'' eller ``undersøke'' en egenskap. Disse formuleringene er ikke testbare: det er ikke mulig å vite om målet er oppfylt eller ikke. Dette gjør at målet med selve eksperimentet er implisitt. Et implisitt mål er uklart, og gjør det senere vanskelig å skrive en god vitenskapelig rapport (se \cref{Rapport}).

\subsection{Kunnskap om god dokumentasjonspraksis}
Undervisningen i eksperimentelle ferdigheter bruker en generell ressurs i Perssons kompendium \cite{Persson2020}, men også mange tilleggsskriv fra de forskjellige faglærene. Eksempler på tilleggs-skriv er heftet til Persson i emnet FY1001 \cite{Persson2020FY1001}. Ingen av kompendiene eller skrivene ser ut til å gi noen tydelige henvisninger til etablerte retningslinjer GDP eller GLP. Praktiske henvisninger til GDP kan være f.eks. Wikipedia \cite{WikiGDP} eller kapittelet ``Good Documentation Practices'' i \cite{Davani2017}. En praktisk henvisning til GLP kunne være f.eks. deler av OECD sine prinsipper \cite{OECD1997}.

Det generelle kompendiet \cite{Persson2020} beskriver en bruk av journal som ikke er vanlig i næringsliv og oppdragsforskning. Prinsippene for dokumentasjon av ideer, protokoller, planer, eksperimentelle oppsett, beregninger og resultatvurderinger anvendes i næringsliv og oppdragsforskning. Imidlertid er det svært vanlig i arbeidslivet at selve dokumentasjonen må være tilgjengelig for flere personer, for eksempel prosjektmedarbeidere. Dette sikrer at prosjektets fremdrift er robust og ikke avhengig av at alle er på jobb hele tiden. Derfor er det vanlig å lagre det meste av dokumentasjon i en elektronisk ressurs og ikke i en håndskrevet journal. En elektronisk ressurs kan for eksempel være et elektronisk dokumenthåndteringssystem eller en filserver.
Prinsippet om samtidig datainnsamling står fast i eksperimentelt arbeid. Automatisert datainnsamling er svært vanlig, og kan for eksempel sørge for at data samles på en felles filserver der de er beskyttet mot modifikasjon.

Manuell datainnsamling gjøres også, ofte i håndskrevne eller elektroniske journaler. Det er vanlig i en håndskrevet journal å henvise til dokumentasjonen for protokoller, beregninger, resultatvurderinger og liknende. Kompendiet i TMT4110 gir gode forslag til hvordan studentene skal strukturere tabeller og journal for samtidig innsamling av måledata underveis i eksperimentet. Dermed gir labøvelsene i TMT4110 studentene mye trening i manuell datainnsamling. Faget gir studentene en grunnopplæring i manuell datainnsamling som er relevant og tilstrekkelig for arbeidsgivere. Studieprogrammet har ikke en liknende, veiledet trening i manuell datainnsamling i noen av fysikkfagene.

\subsection{Kjennskap til risikoanalyse}
Kompendiet i faget TMT4110 gir en innføring i FMEA-metodikk for risikoanalyse og viser risikomatriser. Kompendiet gir en innføring i merking av kjemikalier og bruk av material safety data sheets (MSDS). HMS er behandlet som eget punkt i hver labøvelse. Labøvelsene 5--10 i faget TFY4190 gir trening i å gjennomføre en risikoanalyse etter FMEA-metodikk. HMS blir nevnt i røntengdiffraksjonsoppgaven i TFY4220, der studentene skal måle røntgenstrålingen fra instrumentet.
Til sammen gir disse fagene studentene en grunnopplæring i risikoanalyse som er relevant og tilstrekkelig for arbeidsgivere.

\subsection{Kunnskap om vanlig brukte måleteknologier}
% Gis det tilstrekkelig undervisning i tolkning av storedatasett og/eller artificial intelligence i numerikkfagene? (Kari)
Faget FY1001 gir studentene kunnskap om å logge data fra en sensor og bruke sensorprodusentens egne software. Fagene TFY4185 og TFY4190 utvikler disse kunnskapene betydelig og gir kunnskap om bruk av Labview og spesialsydd programvare for sensorer, sensorer, og datalogging. Disse kunnskapene er svært nyttige og direkte anvendbare i mange typer jobber innenfor forskjellige fagfelt.

Faget TFY4220 gir studentene kunnskaper om FTIR-spektrometri, røntgendiffraksjon og TEM. Alle disse teknikkene er vanlige å bruke innenfor forskning og utvikling og i enkelte typer industri. SEM er også mye brukt i flere bransjer i arbeidslivet. Det kan vurderes om studentene som følger studieprogrammet i Teknisk fysikk skulle ha en introduksjon til SEM på samme måte som blir gitt i faget TFY4260.
Faget TMT4110 gir studentene kunnskaper om et bredt spekter av vanlige teknikker i kjemisk laboratoriearbeid. Teknikkene er vanlige å bruke innenfor enkelte typer industri.

Faget TFY4260 gir studentene kunnskaper om cellekultivering, flowcytometri, måling av bioimpedans, farging, merking, og forskjellige mikroskopiteknikker. Mikroskopiteknikker er svært vanlige å bruke i mange typer jobber innenfor forskjellige fagfelt. Farging, merking, cellekultivering og flowcytometri er svært vanlige teknikker innenfor forskning og utvikling i biologi, medisin og biofysikk.

Faget TBT4102 gir studentene kunnskaper om kromatografi og elektroforese. Begge disse teknikkene er vanlige å bruke innenfor forskning og utvikling og i enkelte typer industri.

To av laboratorieoppgavene i faget FY1001 og alle laboratorieoppgavene i fagene FY1003, TFY4163 og TFY4165 har preg av å være demonstrasjoner som hjelper studentene å forstå teorien de leser. Disse laboratorieoppgavene ser ut til å støtte teoriundervisningen og ikke gi studentene faglige kunnskaper utover det som også formidles gjennom forelesningene.

\subsection{Kunnskap om eksperimentdesign}
Med to unntak er undervisningen i eksperimentelle ferdigheter organisert som en trinnvis øvelse som studentene ledes gjennom. Selv om studentene selv skal gjennomføre det praktiske arbeidet, legger undervisningen ikke til rette for at studentene selv skal finne et egnet måleoppsett eller skrive protokoll. Unntakene er fagene TFY4190 og TBT4102. I TFY4190 skal studentene planlegge, og gjennomføre et eksperiment og skrive rapport, og i TBT4102 skal studentene skrive protokoll, gjennomføre eksperimentelt arbeid, og skrive rapport.

Oppgaver med trinnvise instruksjoner kan egne seg godt for å skape forståelse av en fysisk egenskap eller mekanisme. Likevel, i arbeidslivet er det sannsynlig at mange studenter forventes å kunne omsette komplekse oppgaver eller problemstillinger til et eksperiment som kan hjelpe til å ta en beslutning, komme videre med oppgaven eller kaste lys over problemstillingen.

Det er omfattende å designe et eksperiment fordi det må gjøres mange vurderinger før protokollen er endelig. Likevel, det kunne tenkes at en del momenter ved design av et eksperiment kunne bli en del av allerede eksisterende laboratorieoppgaver. Dette ville presentere studentene for hvilke vurderinger som kan være viktige når et eksperiment skal designes, og gi noe trening i strategier for å designe et eksperiment.

Vurderinger som kunne være aktuelle å legge inn som en del av en eller flere laboratorieoppgaver:
\begin{itemize}
  \item På forhånd bestemme hvilken nøyaktighet måleresultatet må ha for at resultatet skal ha verdi. Eksempel: hvor nøyaktig må eksperimentet ``bestemme vekt på koffert'' være for at en er sikker på å ikke måtte betale for overvektig bagasje på flyet.
  \item Vurdere alternativer for eksperimentelle metoder: hvilke kan og hvilke kan ikke gi den nødvendige nøyaktigheten i målingene. Eksempler: må dimensjoner måles med mikrometerskrue eller linjal? Hvor mange desimaler må en vektmåling ha? Hvilken spesifikasjon må en trykkmåler ha for å kunne bestemme høyden på Nidarosdomen med usikkerhet $\pm\SI{1}{\meter}$? Feilanalyse er tema flere ganger i MTFYMA: i \cite{Persson2020}, i Cavendish’ eksperiment i FY1001, i oppgaven om overflatespenning i TFY4163, i oppgaven om Einsteintemperatur i TFY4165, og i oppgaven om røntgendiffraksjon i TFY4220. I laboratorieoppgavene gjøres imidlertid feilanalysen etter at eksperimentet er gjort. I arbeidslivet vil det være naturlig å vurdere nøyaktighet i eksperimentet på forhånd for å kunne avgjøre om den eksperimentelle metoden kan brukes eller ikke.
  \item Sammenlikne ressursbruk ved alternative eksperimentelle metoder: hvor mye tid og penger trengs for å gjennomføre eksperimentet?
  \item Bestemme det nødvendige antall paralleller/replikater som må gjøres av et eksperiment for å få størrelsen på tilfeldig feil til å bli ``lav nok'' og bestemme hva som er ``lavt nok''.
  \item Finne om det er én verdi i et beregnet uttrykk som dominerer usikkerheten i måleresultatet. Vurdere om det er et annet måleutstyr som bør skiftes ut for å bedre nøyaktigheten. Eksempel: en mer nøyaktig vekt, en sensor med høyere oppløsning.
  \item Se effekten av å bruke kalibrerte måleinstrumenter og mulig konsekvens av ikke-kalibrerte måleinstrumenter. Ingen av laboratorieoppgavene gir en demonstrasjon av mulig konsekvens av å bruke ikke-kalibrerte måleinstrumenter, selv om flere oppgaver nevner at måleutstyret er kalibrert.
  \item Se effekten av å gjennomføre en ``system suitability test'', f.eks. å se at spekteret til en kjent prøve er innenfor fastsatte krav før utstyret brukes til å måle en ukjent prøve, eller måle en kjent vektstandard før en ukjent prøve.  En ``system suitability test'' gjøres med silisiumpulver i røntgendiffraksjonsoppgaven i TFY4220. Oppgaven om Einsteintemperaturen til aluminium i TFY4165 nevner behovet for vatring og tarering av vekt. Det kan vurderes om konseptet ``System suitability test'' bør introduseres i et tidligere årskurs slik at alle studentene på MTFYMA blir kjent med det.
\end{itemize}

\section{Behov i framtidas samfunn: Generiske ferdigheter {\color{red}Alle, forslag er at vi ferdigstiller dette etter at vi er fornøyde med de andre avsnittene}}
\label{Behov}
Arbeidgivere vil forvente og vil ha nytte av at uteksaminerte studenter fra MTFYMA har faglig kunnskap innenfor fysikk, kjemi, matematikk, statistikk og programmering.
Arbeidgivere vil forvente at studentene har en solid teoretisk kunnskapsbase og at de kan gjøre avanserte beregninger, både analytiske og numeriske.

To studieprogram!

\subsection{Sammenfatte resultater av en undersøkelse i en rapport}
I arbeidslivet er det svært nyttig å kunne skrive en vitenskapelig rapport.
Strukturen i en vitenskapelig rapport er egnet til å dokumentere en prosess på en tydelig og ryddig måte som gjør det mulig å gjenskape hva som ble gjort og vurdert.
Den vitenskapelige rapporten har som formål å gjøre det tydelig for leserne hvilket problem som er behandlet, hvilken metode som er brukt til å behandle problemet og hvilket resultat metoden ga.
Videre skal rapporten gi en vurdering av resultatet og en konklusjon eller anbefaling.

Den vitenskapelig rapporten er mye brukt i industri og oppdragsforskning, selv om resultatene ikke nødvendigvis skal bli en vitenskapelig artikkel.
Strukturen i en vitenskapelig rapport er godt egnet som skriftlig dokumentasjon i mange sammenhenger, for eksempel til å presentere valg av leverandør for et innkjøp, valg av arkitektur for en IT-applikasjon, valg av teknisk løsning for et problem i industriproduksjon, resultater av et vitenskapelig eksperiment.
De formelle kravene til en vitenskapelig rapport er gode og nyttige verktøy for å sikre lesbarhet og sporbarhet i argumentasjonen.
Noen eksempler på formelle krav er å ha figurtekst under figurer, tabelltekst {\color{red}(som oftest)} over tabeller, å referere til alle figurer og tabeller i teksten, og å skrive referanseliste riktig.

En vitenskapelig rapport må også diskutere og tolke resultatene i lys av godt beskrevne antagelser,  forutsetninger og betingelser, og sammenlikne dem med eventuelle resultater som er oppnådd av andre.
Dette plasserer undersøkelsen/eksperimentet i sammenheng med allerede etablert kunnskap.
Dette er igjen viktig for å vurdere resultatenes legitimitet.
Til sist må metodikken som er benyttet for å oppnå resultatene beskrives tilstrekkelig detaljert til at andre kan reprodusere rapportens resultater.
Å kunne skrive en vitenskapelig rapport er derfor en generisk ferdighet som er etterspurt i arbeidslivet nesten uavhengig av fagfelt og arbeidsoppgaver.

\subsection{Erfaring med å jobbe både i prosjektform og selvstendig}
Mange oppgaver i arbeidslivet er organisert som et prosjekt med en prosjektleder og prosjektmedarbeidere.
Dette gjelder også eksperimentelt arbeid.
Det er nyttig for arbeidslivet at studentene får noe erfaring med å utføre eksperimentelt arbeid sammen med andre i prosjektform.
Når et eksperiment er organisert som et prosjekt med flere medarbeidere settes det flere krav til dokumentasjonspraksis hos hver enkelt medarbeider.

Samtidig som prosjektet er en vanlig arbeidsform er arbeidsgivere også avhengig av at arbeidstakerne kan jobbe selvstendig.
Evne til selvstendig arbeid kan bety å ha strategier for å kunne angripe en kompleks problemstilling på en konstruktiv måte.

Å jobbe i prosjekt, der det totale arbeidsomfanget er delt opp i arbeidspakker/delprosjekter krever at den/de som har planlagt prosjektet klarer å se hvordan de ulike arbeidsoppgavene støtter opp om og avhenger av hverandre.
Dette stiller krav til systematisk tenking og analyse av problemstillingen som skal løses og vil være en ettertraktet ferdighet i arbeidslivet.

\subsection{Formulere et forskningsspørsmål eller en testbar hypotese}
Motivasjonen for et eksperiment kan ha forskjellige kilder, for eksempel:
\begin{itemize}
  \item Å teste en ny teori.
  \item Å ha en kunde som betaler.
  \item Å velge en optimal teknologi for noe som skal utvikles.
  \item Å finne en løsning på et problem som har oppstått.
\end{itemize}

Hvilken motivasjon som er vanlig kan variere mellom sektorer og fagfelt.
I mange situasjoner der behovet er å velge optimal teknologi eller løse et eksisterende problem vil det være mest kostnadseffektivt å gjøre eksperimenter fremfor å utvikle nye teorier.
Det kan kreve store ressurser å utvikle teori for komplekse situasjoner som for eksempel fukting på overflater med varierende ruhet eller spredning av lys på en samling partikler med ujevn form og varierende størrelse.
Eksperimentelle resultater kan gi den nødvendige innsikten direkte.

Når eksperimenter skal brukes som grunnlag for å velge teknologi eller løse eksisterende problem er det essensielt at eksperimentet bygger på en klart formulert problemstilling (forskningsspørsmål) eller en testbar hypotese. Uten et klart forskningsspørsmål er det usikkert hvorfor eksperimentet skal gjennomføres. I arbeidslivet er det vanligvis bare eksperimenter med en klar nytteverdi som blir gjennomført. Derfor er det nyttig for arbeidslivet at arbeidtakerne har trening i å omsette et komplekst spørsmål i en eller flere forskningsspørsmål eller klart formulerte hypoteser.

\subsection{Kunne redegjøre for antagelser som ligger til grunn for modell/ eksperiment-utforming}
Ved anvendelse av matematiske modeller og sammenlikning med eksperimentell data er det essensielt å kunne redegjøre for hvilke antagelser som ligger til grunn for modellen og eksperimentet og hvilken betydning antagelsene vil ha for tolkningen av resultatene.
Dette stiller krav til innsikten i problemstillingen som studeres og vil kunne brukes for f.eks. å kritisere/forbedre modellen i de fall det er stort avvik mellom modell og eksperiment.

\subsection{Kunnskap om god dokumentasjonspraksis}
I en jobbsituasjon må eksperimentelt arbeid dokumenteres. Det er nyttig for arbeidslivet at arbeidstakerne har kjennskap til retningslinjer for god dokumentasjonspraksis (GDP), god laboratoriepraksis (GLP), og dataintegritet (DI) og vet hvor de kan finne oppdateringer av disse retningslinjene.

Studenter bør være kjent med Laboratory Information Management System (LIMS) systemer og hva deres formål er og ikke er. 
%Jeg jobber til dels med Labvantage LIMS og jeg kan ikke redegjøre for hva dets formål ikke er :) Jeg lurer på hva vi mener med dette? Kari

\subsection{Kunnskap om god kodeutviklingspraksis}
I et stadig mer digitalisert næringsliv bør studenter være kjent med verktøy for versjonskontroll av kildekode og rutiner for testing (enhetstesting) og godkjenning av ny eller forbedret kode. Hvorvidt studenter bør være kjent med elementer av sikker kodedesign samt tungregning bør vurderes. I distribuerte kodeutviklingsprosjekter (i tid eller rom) bør studentene være kjente med ulike  arbeidsformer som benyttes i industrien, samt metoder for ``automatisk'' dokumentasjon av kode (f.eks. \verb+doxygen+).

\subsection{Kjennskap til risikoanalyse}
Mange arbeidsplasser og bedrifter bruker standardiserte metoder for risikoanalyse, f.eks. Failure Mode and Effect Analysis (FMEA). Det er nyttig for arbeidsgivere at nye ansatte har kjennskap til konseptet risikoanalyse, og gjerne kjennskap til en metodikk for å gjennomføre risikoanalyse.

\section{Behov i framtidas samfunn: Digital kompetanse og beregningsorientert fysikk {\color{red} Morten+alle}}
\subsection{Kompetanse i naturvitenskapelige beregninger}

Naturvitenskapelige beregninger betyr å løse vitenskapelige og teknologiske problem ved hjelp av datamaskiner. 
Med kompetanse i naturvitenskapelige beregninger sikter vi til en samlende kompetanse i det som på engelsk kalles for Computational Science og Data Science\footnote{Vi har valgt henhodsvis {\bf beregningsvitenskap} for det engelske begrepet computational science og {\bf datavitenskap} for det engelske samlebegrepet Data Science.}. For et studium i fysikk vil dette inkludere metoder kopla opp mot analyse av data (behandling av data og statistisk analyse), metoder fra numerisk analyse, analytiske metoder, symbolske beregninger og mer. 
Naturvitenskapelige beregninger, eller bare beregninger, dekker dermed numeriske beregninger, symbolske beregninger og analytiske beregninger. Slik vi bruker ordet {\bf beregninger} i teksten her inkluderer det dermed metoder fra både beregningsvitenskap og datavitenskap, symbolske og analytiske beregninger.  Beregninger er sentrale element i et utdanningsløp i fysikk og legger grunnlaget for en djupere innsikt om ulike fysiske problem og vitenskapelig tenkning. Samtidig spiller beregninger en sentral rolle i all moderne naturvitenskap og teknologisk utvikling. 
Kompetanse i beregninger inkluderer blant annet

\begin{itemize}
  \item utledning, verifsering og validering av kjente algoritmer
  \item en forståelse av en gitt algoritmes begrensninger
  \item kjennskap til sentrale algoritmer i naturvitenskap
  \item forstå hvordan ulike algoritmer kan brukes i naturvitenskapelig modellering
  \item utvikle en algoritmisk tenkning for å tilegne seg en djupere innsikt om naturvitenskapelige og teknologiske problem
\end{itemize}

\subsection{Sentrale element i beregningskompetanse}
Kompetanse om beregninger består i å identifisere et gitt problem som et spesielt tilfelle av en abstrakt klasse av problemer, å kunne sette opp generelle løsningsmetoder for denne klassen av problemer, og anvende en generell metode for det spesifikke problemet (databehandling, beregninger med penn og papir, symbolsk beregninger, eller
numeriske beregninger med ferdiglaget og / eller egenskrevet programvare). Et slikt 
generisk syn på problemer og metoder er spesielt viktig for
å kunne forstå hvordan en bruker tilgjengelig, generisk programvare for å løse et gitt problem.

Kompetanse om beregninger er et sentralt element
i vitenskapelig problemløsning, fra grunnleggende utdanning og forskning til
i det vesentlige nesten alle avanserte problemer i vårt moderne
samfunn. Beregningskompetanse  utvider mengden verktøy som er tilgjengelig for studenter og
forskere utover klassiske verktøy og gir mulighet for en mer generisk
håndtering av problemer. Å fokusere på algoritmiske aspekter resulterer som regel i
djupere innsikt om vitenskapelige problemer.

Dagens prosjekter innen vitenskap og industri har en tendens til å involvere større team. Verktøy for pålitelig samarbeid må derfor mestres (f.eks. programvare for versjonskontroll, automatiserte dataeksperimenter for reproduserbarhet, programvare og metodedokumentasjon og mye mer).

\subsection{Læringsutbytter }
Etter avsluttet studieløp er følgende kompetanser, ferdigheter og kunnskaper sentrale:

\begin{itemize}
  \item \textbf{Kunnskaper}:
    \begin{itemize}
      \item har en djup forståelse for den vitenskapelige metoden og sentrale beregningsmetoder på et avansert nivå, som innebærer at kandidaten
        \begin{enumerate}
          \item har evnen til å forstå avanserte vitenskapelige resultat i nye felt
          \item har kjennskap til sentrale algoritmer i datavitenskap og beregningsvitenskap.
          \item har en djup forståelse av sentrale numeriske metoder som er brukt i naturvitenskap
          \item kan utvikle og anvende avanserte numeriske metoder i vitenskapelige og teknologiske problemer
          \item kan vurdere og analysere alle deler av oppnådde vitenskapelige og teknologiske resultater
          \item kan presentere skriftlig og muntlig resultatene i form av rapporter og /eller vitenskapelige artikler
          \item kan foreslå nye hypoteser og løsningsalternativ
          \item kan generalisere matematiske algoritmer og anvende de på nye vitenkapelige og teknologiske problem
          \item kan gjøre resultatene reproduserbare  og har kjennskap til sentrale verktøy for versjonskontroll.
        \end{enumerate}

        \noindent
      \item har en djup forståelse og kunnskap om vitenskapelig arbeid, som innebærer at kandidaten
        \begin{enumerate}
          \item kan utvikle hypoteser og foreslå måter å teste disse
          \item kan bruke analytiske, numeriske og eksperimentelle metoder og verktøy for å teste vitenskapelige hypoteser
          \item kan presentere metoden og resultatene i henhold til god vitenskapelig praksis
        \end{enumerate}

        \noindent
    \end{itemize}

    \noindent
  \item \textbf{Ferdigheter}:
    \begin{itemize}
      \item har en djup foståelse for hva beregninger betyr og 
        \begin{enumerate}
          \item kjenner til sentrale algoritmer i datavitenskap og beregninsgvitenskap og hvordan disse kan implementeres med eksisterende software for å løse vitenskapelige problem
          \item forstår betydningen av feil fra approksimasjoner og hva som kan gå galt med ulike algoritmer
          \item kjennskap til computer algebra system og hvordan det kan bukes til klassiske matematiske operasjoner
          \item har gode ferdigheter i høynivå programmeringsspråk som (Python eller liknende)
          \item har gode ferdighter i kompilerte programmeringsspråk som feks C++ eller liknende.
          \item kjenner til verktøy for å analysere og debugge software
          \item har kjennskap til test rammeverk og prosedyrer
          \item er i stand til å vurdere resultatene  og mulige feilkilder
          \item kan utvikle algoritmer og software for kompliserte vitenskapelige og teknologiske problem individuelt og i samarbeid med andre
          \item kjenner til verktøy for å kunne gjøre resultater reproduserbare
        \end{enumerate}
    \end{itemize}
\end{itemize}

\subsection{Kjennskap til sentrale metoder i datavitenskap og beregningsvitenskap}
Vi ser for oss at ulike beregningsalgoritmer presenteres og brukes på et systematisk og koherent vis gjennom hele studieløpet, i tett samarbeid med andre institutt. Et synkronisert studieløp letter også introduskjonen og bruken av ulike beregningsmetoder i et gitt fysikkfag. 
Vi anbefaler at følgende sentrale algoritmer har blitt presentert og diskutert gjennom et studieløp (det er opp til institusjonen å spesifisere de eksplisitte metodene)
\begin{itemize}
  \item Sentrale metoder i linær algebra, fra direkte metoder til iterative
  \item Sentrale metoder for å løse partielle og ordinære differensiallikninger
  \item Numerisk integrasjon
  \item Stokastiske metoder og prosesser, Markov kjeder og Monte Carlo simuleringer 
  \item Sentrale verktøy i statistisk dataanalyse, med maskinlæringsmetoder som linær og logistisk regresjon og neurale nettverk
  \item Sentral software i high-performance computing 
\end{itemize}

\section{Behov i framtidas samfunn: Eksperimentell kompetanse {\color{red} Dag + alle}}
\subsection{Generelt om eksperimenter i fysikkundervisningen}
Man kan dele motivasjonen for laboratorieundervisning opp i to hovedbolker:
\begin{itemize}
    \item Å lære å måle og å drive eksperimentelt arbeid.
    \item Å bidra til innlæringen av utvalgte emner i
    fysikken.
\end{itemize}

Personlige erfaringer i arbeidsgruppa samt inntrykk fra intervju med studenter ved studieprogrammet viser at det er utfordrende å inkorporere eksperimenter på en motiverende og fruktbar måte.
I tillegg til at eksperimentelle aktiviteter skal bidra til at studentene utvikler grunnleggende laboratorieferdigheteter og får en dypere innsikt i grunnlaget for teoretisk fysikk, er det viktig at studentene lærer at eksperimenter har en egenverdi da utviklingen av de fleste vitenskapelige teorier starter med observasjoner/eksperimenter som ikke kan forklares med eksisterende modeller/teori.  
Det virker avgjørende at studente forstår hvorfor eksperimenter skal utføres og hvordan disse støtter opp om teori og beregningsktiviteter i ikke bare det aktuelle kurset, men hele fysikkstudiet.
Der læringsutbyttet skal være generiske laboratorieferdigheter (e.g. måleteknikk eller bruk av spesielle apparaturer eller metoder), bør det være tydelig for studentene.
Der eksperimentene skal belyse eller eksemplifisere spesielle fysiske observasjoner/fenomener virker det formålstjenelig at eksperimentet motiveres gjennom forelesning og teoriøvinger forut for laboratorieaktiviteten.

\paragraph{Måling og eksperimentelt arbeid}
Hva slags kunnskaper, holdninger og ferdigheter er det studentene bør tilegne seg? Sentrale dyder som presisjon, nøyaktighet og ærlig dokumentasjon er grunnleggende i vitenskapelig eksperimentelt arbeid samt de følgende kunnskapene og ferdighetene:
\begin{itemize}
  \item Måling av grunnleggende fysiske størrelser som tid, lengde, masse, kraft, temperatur, strøm og spenning. Lære og prøve ut forskjellige prinsipper for å måle disse størrelsene. Lære å vurdere hva som begrenser nøyaktigheten til målingene. Lære om historisk utvikling av grunnlaget for målenhethene våre (SI-enhetene) og hvilke fysiske prinsipper som i dag definerer enhetene og hvilke målinger som ligger til grunn for sentrale fysiske konstanter.
  \item Kilder til feil i målinger og hvordan kvantifisere disse.
  \item Behandling av måledata. Statistisk behandling, tilpassing av modeller til måledata med støy.
  \item Lære å bruke noen måleinstrumenter som er mye brukt i fysikk
  \item Dokumentere planlegging, gjennomføring, observasjoner, måledata og dataanalyse i en sporbar sammenheng frem til endelig rapport.
  \item Eksperimentelle strategier. Det fleste laboratorieoppgaver er laget for å lære studentene det læreren allerede kan. Teorien er kjent og hypotesen skal ``testes''. Dette kan brukes til å undervise eksperimentdesign. I komplekse og tverrfaglige problemstillinger finnes det ikke en ferdig teori, ingen klare hypoteser. Eksperimenter brukes da ofte til systematisk ``eksperimentering'' for å finne sammenhenger og styre utviklingen av teoretiske modeller. 
\end{itemize}

\paragraph{Lære fysikk}
De følgende læringsmålene er ofte viktige i laboratorieundervisningen:
\begin{itemize}
  \item Å illustrere teori som gjennomgås i forelesninger og øvinger. Labkurs kan brukes for å støtte opp under teoretiske kurs som går paralellt i tid eller ta opp igjen temaer som har vært berørt tidligere i utdanningen.
  \item Å lære om spesifikke fysiske lover/fenomener som
    ikke dekkes i de teoretiske kursene.
  \item Å bygge intuisjon. Ikke alle forstår verden som konkrete eksempler på abstrakte sammenhenger. Det er kanskje fysikeres ønske å kunne tenke på verden slik, men man kommer ikke dit uten å lære det. En del kunnskap bygger på generalisering av erfaring. Ved å ta sansene i bruk og ikke bare den abstrakte tenkningen øker man erfaringsrommet og gir studentene mulighet til å “forstå” på et annet vis enn ved å utlede formler først.
\end{itemize}

\subsection{Kunnskap om eksperimentdesign}
Et eksperiment bør være designet slik at det er egnet til å teste en gitt hypotese.
Det må være en protokoll eller plan for eksperimentet.
Et eksperiment må isolere den eller de egenskapene som vil gjøre det mulig å bekrefte eller avkrefte hypotesen.
Det kan være vanskelig å designe et eksperiment som tester en gitt hypotese.

Det er nyttig for arbeidsgivere at arbeidstakerne har kunnskap om framgangsmåter som kan brukes for å designe et eksperiment, og at de har noe trening i å designe et eksperiment som gir den nøyaktigheten som er nødvendig for å kunne teste hypotesen. I en jobbsituasjon vil det vanligvis være begrensete ressurser som kan brukes på et eksperiment, f.eks. tid og/eller penger.
Det er nyttig for arbeidsgivere at en arbeidstaker har opplæring i følgende ferdigheter:
\begin{itemize}
  \item Bestemme hvilken nøyaktighet et eksperimentet må ha for å kunne avkrefte eller bekrefte en hypotese med en bestemt grad av sikkerhet.
  \item Diskutere alternative måleteknikker, identifisere hvilke(n) som er egnet, og finne egnet måleutstyr og oppsett.
  \item Vurdere antall målinger/replikater som må finnes for å for å ha en akseptabel verdi for tilfeldig feil, beregne feilestimat for de målte verdiene, og vurdere om noen dominerende feilkilder kan/bør minkes.
  \item Sørge for at eksperimentoppsettet vil gi gyldige data: at måleutstyr er kalibrert og at oppsettet tilfredsstiller en ``system suitability test'' før det brukes til å samle inn data.
  \item Bruke automatiserte løsninger for datainnsamling og evt. dataanalyse.
  \item Avpasse eksperimentet etter begrensninger på tid og kostnad og ønsket nøyaktighet på resultatet.
\end{itemize}

\subsection{Kunnskap om vanlig brukte måleteknologier}
Innenfor eksperimentelle ferdigheter vil arbeidsgivere ha nytte av at arbeidstakerne er kjent med de aller vanligste eksperimentelle teknikkene som brukes innenfor fagområdet.
I tillegg vil arbeidsgivere ha nytte av at arbeidstakerne har kunnskaper som setter dem i stand til å sette opp automatisk datainnsamling fra et måleoppsett.
Arbeidsgivere forventer ikke at nylig uteksaminerte studenter har detaljkunnskap om den nyeste teknologien og metodikken innenfor et smalt fagfelt eller bruksområde.
I næringsliv og anvendt forskning forventes det at alle ansatte har grunnkunnskaper som setter dem i stand til å forstå og anvende ny teknologi og metodikk ettersom den blir tilgjengelig og er til nytte i arbeidsoppgavene som skal utføres.
Innen akademia forventes det at ansatte i tillegg selv er i stand til å utvikle ny teori, teknologi og/eller metodikk.

\section{Behov i framtidas samfunn: Integrasjon av beregninger, teori og eksperiment i studieløpet {\color{red} Paal + alle}}

\section{Bærekraftsperspektiv, behov i framtidas samfunn {\color{red} Sverre+alle}}
Norske og internasjonale bedrifters aktiviteter måles i økende grad mot FNs 17 bærekraftsmål for å utrydde fattigdom, bekjempe ulikhet og stoppe klimaendringene innen 2030\cite{FNsustgoals}.
NTNU har forpliktet seg til å arbeide med bærekraftsmålene gjennom sin visjon \emph{Kunnskap for en bedre verden} og det tematiske satsningsområdet \emph{Bærekraft} \cite{NTNUStrategi,NTNUBaerekraftMaal,NTNUBaerekraft}

Norsk Standard inneholder en rekke verktøy som er egnet til å systematisere arbeidet med bærekraftsmålene innen ingeniørfaglige bransjer som f.eks. bygg og anlegg, petroleum, IKT, helse og matproduksjon.
Noen generelle standarder er oppsummert på Standard Norges web-sider\cite{StandardNorge}.
Framtidens ledere, forskere og ingeniører må påregne å måtte ha et aktivt forhold til disse.
Det er hevet over tvil at fysikere vil ha en avgjørende rolle i oppnåelsen av FNs bærekraftsmål.

Der Norsk standard har en naturlig plass i de bransjerettede ingeniørutdanningene, er det uklart hvordan bærekraftsmålene naturlig kan introduseres i fysikkstudiet.
Som et minimum bør det forventes at bærekraftmålene introduseres og arbeides med i fellesemnene (Ex.phil., Områdeemnet og Eksperter i Team)\cite{NTNUFellesEmner}.
I tillegg kan man se for seg at tematikken kan belyses gjennom prosjektoppgaver og semesteroppgaver i øvrige fag.

\section{Etiske betraktninger {\color{red} Morten+alle}}
Her vektlegger vi vitenskapsetiske moment som lett kan implementeres i et studieløp ved hjelp av moderne software som for eksempel versjonskontroll software.
\begin{itemize}
  \item {\bf Reproduserbarhet av vitenskapelige resultat}. Ved hjelp av versjonskontroll software som {\bf git} og liknende samt lagringsplass via servere som GitHub, GitLab, Bitbucket og andre, er det lett å opprettholde en historikk av vitenskapelige resultat. Her kan en også lagre resultat av bestemte kjøringer og testresultat som letter reproduserbarhet av vitenskapelige resultat. Å gjøre kode og resultat tilgjengelig, med god dokumentasjon, gjør det mulig for andre å teste og kunne reprodusere publiserte resultat. Kunnskap om versjonskontroll software bor være inkludert i studieløpet.
  \item {\bf Korrekt sitering}. Biblioteksverktøy  av typen Mendeley og andre, gjør det enklere å sitere vitenskapelige arbeider. Studieløpet bør dermed utvikle en god kultur i å sitere andres arbeid når vitenskapelige rapporter ferdigstilles, enten i kurs eller som en endelig avhandling eller prosjektarbeid.
  \item {\bf Regler ved samarbeid}. Studentene oppfordres til å utvikle samarbeidsprosjekter og jobbe i lag. Å jobbe som et lag er sentrale ferdigheter og evner som er sentrale i et moderne arbeidsliv. Software for vitenskapelig samarbeid og etiske retningslinjer for samarbeid bør integreres i studieløpet. Dette vil spesielt gjelde kurs der prosjektarbeid spiller en sentral rolle.
  \item {\bf Vitenskapelig juks og plagiering}. Studieløpet bør utvikle en grunnleggende positiv holdning til deling av vitenskapelige resultat via for eksempel mulighetene  som servere av typen GitHub, GitLab og andre gir. Dette kan åpne for juks og plagiering siden mange resultat vil være fritt tilgjengelige. Studieløpet bor derfor ha en fortløpende diskusjon om temaer som juks og plagiering og deres følger. Software som {\bf Grammalry} og andre bør integreres i slike diskusjoner. 
\end{itemize}

\section{Sammenlikning med laboratorieundervisning ved andre læresteder {\color{red} Kari + alle}}
For å se på hvordan undervisningen i eksperimentelle ferdigheter er organisert ved andre europeiske læresteder, plukket vi ut tre eksempler. Vi valgte École Polytechique Féderale i Lausanne (EPFL), Eidgenössische Technische Hochschule i Zurich (ETHZ) og Technische Universität München (TUM). Alle disse tre er blant de ti øverste på QS World University Rankings innenfor «Engineering and Technology» og har studieløp for Bachelor- og mastergrad i fysikk/matematikk \cite{ETHZprog}, \cite{EPFLprog}, \cite{TUMprog} som er sammenliknbare med MTFYMA ved NTNU.  I tillegg følger en beskrivelse av hvordan numeriske beregninger har blitt integrert på et systematisk og helheltig vis ved Universitetet i Oslo siden 2003 via {\bf Computing in Science Education} prosjektet. Planene for en systematisk integrasjon av teori, eksperiment og beregninger ved Universitetet  i Oslo er også beskrevet.

Vi fikk et møte på zoom med lederne for laboratorieundervisningen ved alle disse tre lærestedene. Deltakerne på og tidspunktene for møtene står i listen nedenfor.
\begin{itemize}
  \item TUM: Andreas Hauptner og Martin Sass, Paal Skjetne, Dag Dysthe, Magnus Lilledahl, Ingve Simonsen, Jonas Persson, Kari Schjølberg-Henriksen, avholdt 20. november 2020. 
  \item ETHZ: Alexander Eichler, Andreas Eggenberger og Martin Kroner, Paal Skjetne, Dag Dysthe, Sverre Johnsen, Magnus Lilledahl, Ingve Simonsen, Jonas Persson, Kari Schjølberg-Henriksen, avholdt 20. november 2020. 
  \item EPFL: Daniele Mari, Ingve Simonsen, Kari Schjølberg-Henriksen, avholdt 23. november 2020. 
\end{itemize}

Ved ETHZ er det 155 studenter i 3. semester og 140 studenter i 5. semester. Ved TUM er det 200 studenter på backelornivå og 200 på masternivå, altså ca 130 grupper som skal ha laboratorieundervisning. Ved EPFL er det 120 studenter i 2. årskurs, 90 studenter i 3. årskurs, og 80 studenter på masternivå. 

\subsection{Læringsmål for laboratorieundervisningen}
Laboratorielederne ved alle de tre lærestedene understreket at det viktigste læringsmålet med laboratorieundervisningen er å mestre laboratoriearbeid. Dette innebærer evne til å velge mellom og bruke forskjellige måleinstrumenter, se fordeler og ulemper ved forskjellige eksperimentelle oppsett, kunnskap om kalibrering, skrive protokoll, innehnte data, og skrive rapport. Hovedmålet med undervisningen er at studentene skal få kunnskaper som setter dem i stand til å gjøre egne eksperimenter senere. 

I de første 3 – 4 semestrene er det ikke et mål at studentene skal lære ny fysikk gjennom laboratorieundervisningen. Ved ETHZ og EPFL bygger eksperimentene på fysikk som er godt kjent for studentene, og til dels på fysikk de lærte på videregående skole. Ved TUM er det et mål med undervisningen å repetere fysikken som ble undervist forrige semester. De eksperimentelle temaene er forskjøvet slik at eksperimentene i mekanikk gjennomføres semesteret etter at emnet ble forelest. 
Læringsmålene endrer seg fra ca. 5. semester ved alle lærestedene. Da det blir et uttalt mål også å lære ny fysikk gjennom eksperimentelt arbeid og også et mål å bli kjent med forskjellige fagfelt innenfor fysikken. 

\subsection{Struktur på den teoretiske fysikkundervisningen}
Fysikkundervisningen er lagt opp litt forskjellig ved de tre lærestedene. Ved EPFL og ETHZ undervises det et kurs i faget «Fysikk». Dette kurset går over flere semestre, og dekker mekanikk, termisk fysikk, bølgefysikk, elektrisitetslære, magnetisme, og relativitetsteori. Ved EPFL inngår fluidmekanikk også i kurset «Fysikk», mens ved ETHZ inngår kvantemekanikk og atom- og kjernefysikk. Ved TUM er studiet lagt opp på en måte som likner mer på NTNU sitt studieprogram, med egne kurs i forskjellige fysikkfagene som f.eks. mekanikk, elektromagnetisme, termodynamikk, elektrodynamikk, optikk og atomfysikk. 

\subsection{Struktur og omfang av laboratorieundervisningen}

\subsubsection{Forelesninger i laboratoriearbeid}\label{forel_lab}
Det er stor variasjon mellom lærestedene i hvor mye forelesninger studentene får om laboratoriearbeid. 
Ved EPFL gis det forelesninger hele 2. semester. Forelesningene handler om vitenskapelig metode, god laboratoriepraksis, datainnsamling, og gir en introduksjon til feilanalyse i praksis og formelverk for dette. I parallell med forelesningene skal studentene gjennomføre en praktisk oppgave der de skal bygge sitt eget vakuumkammer med oppsett for å måle motstand. 
Ved ETHZ gis det forelesning 1 time pr uke parallelt med hele labundervisningen. Forelesningen behandler dataanalyse, rapportskriving, en introduksjon til Latex, grunnleggende praktisk statistikk og regresjonsanalyse. Studentene må bestå et separat kurs i HMS før de får tilgang til laboratoriet. 
Ved TUM gis det kun én time forelesning. Den gis i første semester og handler om anvendt statistikk til bruk i analyse av eksperimentelle data. Imidlertid får studentene skriftlig læremateriell som de bruker gjennom hele laboratorieundervisningen. De får et hefte om dagens praksis av god laboratoriepraksis og god dokumentasjonspraksis. Dette heftet inneholder tips til videre litteratur. De får også et hefte om anvendt statistikk og feilanalyse, og et hefte om rapportskriving og en introduksjon til strukturen i en vitenskapelig artikkel. Studentene må ta et nettbasert HMS-kurs før de får tilgang til laboratoriet. 

\subsubsection{Praktiske laboratorieøvinger}
Ved \textbf{EPFL} gjør studentene eksperimenter 2 – 6 semester. Eksperimentet det første studieåret består i å bygge et oppsett for å måle motstand, se kapittel \ref{forel_lab}. Det andre og tredje studieåret jobber studentene i par. Studentene får presentert en stor meny med eksperimenter som dekker mange felt i fysikk. Studentene søker om plass på eksperimentene og får tildelt eksperimenter etter ønske så godt det lar seg gjøre. 
Det andre studieåret skal studentene gjennomføre sju eksperimenter per semester. I hovedsak er disse eksperimentene basert på teori som studentene er godt kjent med fra før. Det tredje studieåret skal studentene gjennomføre tre eksperimenter per semester. Disse eksperimentene blir ikke avsluttet i én økt, men går over fire uker uker. 
Studentene får en «eksperimentmanual» som beskriver teorien bak eksperimentet og selve eksperimentet. Det andre studieåret kommer studentene til ferdige oppsett. Det tredje studieåret er det eksperimentelle oppsettet tegnet og beskrevet i eksperimentmanualen, men det varierer litt fra eksperiment til eksperiment om det er satt opp ferdig eller må settes opp av studentene. 

Ved \textbf{ETHZ} gjør studentene praktiske eksperimenter i 3 – 6 semester. De jobber i grupper på to personer. Studentene får presentert en stor meny med eksperimenter som dekker alle fysikkfeltene som er forelest i de første studieårene. Det andre studieåret gjennomfører studentene en blanding av obligatoriske og valgte eksperimenter. Laboratorielederne sørger for at studentene gjennomfører eksperimenter som til sammen dekker alle fysikkfeltene som tilbys.  Det tredje studieåret har studentene færre obligatoriske og flere egenvalgte eksperimenter. Studentene oppfordres studentene til å fortsette å velge eksperimenter fra varierte felt, men har frihet til å ikke gjøre det.

Eksperimentene er for en stor del ferdige oppsett med en oppskrift som studentene skal følge. Imidlertid holder laboratorielederne på med en fornyelse av eksperimentene. De ønsker å avvikle eksperimenter med ferdige, kompliserte og perfeksjonerte oppsett. I stedet vil de legge til eksperimenter som viser reelle utfordringer med eksperimentelt arbeid, som for eksempel å opplinjere et optisk måleoppsett. Det er nok progresjon i eksperimentene, f.eks. innenfor elektronikk bygger eksperimentene på hverandre og blir mer komplekse etter hvert. ETHZ tilbyr eksperimenter som utføres med simuleringer. Disse eksperimentene er godt egnet for å generere store datamengder og for å illustrere statistikk. 

Studentene får en «eksperimentmanual» som beskriver teorien bak eksperimentet og selve eksperimentet. Laboratorielederne kunne gjerne ønsket å ha flere eksperimenter der studentene får en «boks» med en del utstyr og selv skal sette opp eksperimentet. Imidlertid er dette er ofte vanskelig å få til innenfor tidsbegrensningen på 4 timer.

Ved \textbf{TUM} har studentene praktiske eksperimenter i 2 – 6 semester. I første og andre studieår jobber studentene i grupper på to personer. De skal gjennomføre et forhåndsbestemt program med seks eksperimenter per semester. (https://www.ph.tum.de/academics/org/labs/ap/). Eksperimentene er innenfor fysikken som ble undervist det foregående semesteret. 
I tredje studieår jobber studentene i grupper på tre. De velger seks eksperimenter fra en meny, men fire av eksperimentene må være innenfor deres egne valgte studeretning. Studentene har en diskusjon med labassistenten i forkant av eksperimentet og de får en «eksperimentmanual» med teoribakgrunnen for eksperimentet. Hovedtyngden av alle eksperimentene er å samle eksperimentelle data og sammenlikne dem med en teoretisk verdi. Tredje studieår tilbys det eksperimenter basert på simuleringer. 
Ved TUM er instrumentene som trengs for å gjennomføre eksperimentet tilgjenglige i laboratoriet. For noen eksperimenter må studentene sette opp oppsettet selv, og noen ganger er det ferdig satt opp til studentene. Oppsett med elektriske kretser bygger studentene selv fra bunnen av. Det vurderes å være for krevende for studentene, ta for lang tid og koste for mye om studentene skal designe et oppsett fra bunnen av. 

\subsubsection{Arbeidsbelastning og vurdering}
Ved EPFL får studentene 4 ECTS per semester i semester 2 – 4 og bruker 4 timer pr. uke. Det tredje studieåret får de 8 ECTS og bruker omtrent en dag pr. uke på eksperimentelt arbeid. Arbeidet dokumenteres i rapporter og ved presentasjon. Andre studieår gir studentene en 10 – 15 minutters muntlig presentasjon. Tredje studieår skal studentene både gi en muntlig presentasjon og lage en poster som presenteres på en mini-konferanse. Studentene får karakter på laboratoriearbeidet. Karakteren er basert på rapporter/presentasjon/poster og kvaliteten på det eksperimentelle arbeidet (forberedelser, labarbeid og resultater). 
Studentene gir god tilbakemelding på laboratorieundervisningen i tredje studieår. I andre studieår mener studentene at de får for få ECTS i forhold til hvor mye arbeid rapportene krever. 

Ved ETHZ er det en økende belastning fra 3. til 6. semester med 5 ECTS det 3. semesteret til 8 ECTS det 6. semesteret. Et eksperiment krever et skriftlig forarbeid som skal vises til labassistenten, og det skal skrives en rapport etterpå. Eksperimentene tar 3 – 4 timer i labben det andre studieåret og studentene gjør 8 – 9 eksperimenter per semester. Det tredje studieåret tar eksperimentene opptil 2 dager, men studentene skal bare gjøre fire eksperimenter per semester. Studentene bruker mye mer tid på å skrive rapporten enn på å gjøre selve eksperimentet. 
Det gis ikke karakter, men bestått/ikke bestått. Labundervisningen får blandet tilbakemelding fra studentene. Laboratorieledernes klare vurdering er at labassistenten er helt avgjørende for om studenten liker labundervisningen eller ikke.  

Ved TUM får studentene 5 ECTS i semester 2 – 4 og 6 ECTS per semester det tredje studieåret. Det er vanlig å bruke omtrent 3 timer på det eksperimentelle arbeidet og mye mer tid på rapporten. Labundervisningen er godt likt av studentene, særlig i semester 2 – 4. Det tredje året er labundervisningen godt likt av eksperimentalistene. 

\subsection{Rapportskriving}
Ved EPFL skriver studentene rapporter fra hvert eksperiment i andre og tredje studieår. I andre studieår skriver de individuelle rapporter, det vil si 14 rapporter. Studentene får se eksempler på gode rapporter for å forstå hvordan de skal skrive. 

Ved ETHZ skriver studentene rapporter i par etter hvert eksperiment. Dette betyr 17 rapporter i andre studieår og åtte rapporter det tredje studieåret. Studentene må gjøre selvstendige beregninger og feilanalyse i rapportene. Rapportene rettes og kommenteres av labassistentene. Laboratorielederne forteller at diskusjonen om de konkrete rapportene er en av de viktigste læringssituasjonene i laboratorieundervisningen. 

Ved TUM skriver studentene rapport i par etter hvert eksperiment. De tre første semestrene blir dette 18 rapporter. Alle rapportene gjennomgås og rettes av en labassistent, men tre av rapportene gjennomgås nøye sammen med assistenten. Studentene bruker mye tid på å organisere og skrive rapporten sin, og de trenger også mye tid på å øve seg på å plotte figurer, lage kurvetilpasninger og å presentere statistikk. Også ved TUM anses kommentarene på rapportene som en svært viktig del av laboratorieundervisningen. Labassistentene blir litt strengere med rettingen etter hvert som studentene har gjennomgått flere semestre.  

\subsection{Sammenheng mellom teoretisk fysikkundervisning og praktiske laboratorieøvinger}

Ved ETHZ og EPFL har ikke eksperimentene noen direkte sammenheng med den teoretiske undervisningen. Imidlertid er eksperimentene i 3 – 4 semester bygget på teori som studentene er godt kjent med fra før, slik at de ikke er avhengige av å ha hatt noen bestemte forelesninger først for å ha utbytte av eksperimentet. Ved TUM er laboratorieundervisningen i semester 2 – 4 forskjøvet med et semester slik at eksperimentene bygger på teori som ble forelest forrige semester.

Ved alle lærestedene kan studentene i 5 – 6 semester oppleve at eksperimentet bygger på teori som ennå ikke er forelest. Studentene får utdelt «eksperimentmanualer» som skal gi dem tilstrekkelig teoribakgrunn for eksperimentet. Laboratorieleder ved EPFL understreker at studentene kan oppleve å mangle teoretisk bakgrunn i arbeidslivet også, og at det er greit å forberede studentene på denne situasjonen.

\subsection{Computing in Science Education ved Universitetet i Oslo} 
Mer tekst kommer her asap.

\section{Studentenes opplevelse av nåværende undervisningsopplegg {\color{red}Alle}}

\section{Anbefalinger {\color{red}Alle}}
Basert på arbeidsgivernes behov (\cref{Behov}) og inneholdet i nåværende undervisningsopplegg (\cref{Sammenlikning}), anbefaler den eksterne arbeidsgruppen at ledelsen for studieprogrammene ser nærmere på noen aspekter ved undervisningen i eksperimentelle ferdigheter.
\begin{itemize}
  \item Vurdere å etablere en sentral ressurs med opplæring i gjeldende retningslinjer for skriving av vitenskapelige rapporter og iverksette tiltak som sikrer at denne  ressursen brukes i alle laboratoriekursene. En slik sentral ressurs kan gjerne være utdrag fra bøker og artikler, og ikke nødvendigvis et egenprodusert hefte. En slik sentral ressurs kan også inneholde flere elementer, for eksempel prinsipper for dataanalyse, retningslinjer for eksperimentdesign.  
  \item Vurdere å gi studentene kjennskap til eksterne ressurser om oppdaterte retningslinjer for god dokumentasjonspraksis (GDP) og god laboratoriepraksis GLP). Disse kan gjerne inngå som en del av en sentral ressurs som skal brukes av alle laboratoriekursene.
  \item Vurdere å gi studentene en beskrivelse av gjeldende praksis i journalbruk og journalføring, slik journaler er vanlig brukt i arbeidslivet i dag.
  \item Vurdere å øke antallet vitenskapelige rapporter som studentene skal skrive. 
  \item Vurdere å gjennomføre noen laboratorieoppgaver og/eller vitenskapelige rapporter som individuelt arbeid.
  \item Vurdere å skille tydelig mellom laboratorieoppgaver som skal tjene som demonstrasjon av en fysisk effekt og laboratorieoppgaver som er eksperimenter som skal svare på et forskningsspørsmål eller hypotese. For de siste, vurdere å eksplisitt formulere forskningsspørsmålet/hypotesen.
  \item Vurdere å ta inn elementer av eksperimentdesign og protokollskriving i laboratorieoppgavene. Slike elementer kan være å vurdere ulike alternative eksperimentoppsett med tanke på nøyaktighet, tidsbruk og kostnader, forhåndsberegning av måleusikkerhet og å fastsette antall replikater som er nødvendig, og å vurdere behovet for kalibrering av instrumenter og bruk av ``system suitability tests'' i et eksperimentelt oppsett.
\end{itemize}

\appendix
\section{Mandat for ekstern arbeidsgruppe for evaluering av studieprogrammene MTFYMA og BFY}
\subsection{Bakgrunn}
I henhold til NTNUs system for kvalitetssikring skal alle studieprogram evalueres hvert femte år. Arbeidsgruppen er en del av en slik periodisk evaluering. Fra høsten 2019 innførte Institutt for fysikk såkalte ferdighetsstrenger innen 1) Numerisk fysikk og programmering (NP) og 2) Eksperimentell fysikk (XF). Hensikten er å få til en mer planmessig utvikling av studentenes ferdigheter innenfor disse områdene. 

\subsection{Mål}
Studieprogrammene ønsker å kartlegge eksisterende aktivitet innenfor disse områdene, samt få en ekstern evaluering, for å bruke dette som grunnlag for videreutvikling av ferdighetsstrengene.
Vi ønsker spesielt å få vurdert
\begin{itemize}
  \item I hvilken grad imøtekommer utdanningene samfunnets behov innenfor beregningsorientering og eksperimentell fysikk.
  \item Hvordan sammenlikner den eksisterende og den planlagte utdanningen med hvordan dette adresseres på andre læresteder (både i Norge og internasjonalt).
  \item Er det noe spesielt som mangler innenfor disse områdene i et bærekraftperspektiv?
\end{itemize}

\subsection{Organisasjon}
Den eksterne arbeidsgruppa består av minst 6 representanter, 3 fra hvert av fagområde (beregningsorientering og eksperimentell fysikk). Det skal være minst en fra akademia og en fra industri innen hvert fagområde.
Ett av medlemmene i gruppa utpekes som leder og er ansvarlig for gjennomføring i henhold til mandatet.

\subsection{Gjennomføring/fremdrift/rapportering}
Studieprogrammene gjennomfører først en intern kartlegging av nåværende innhold i emnene relatert til ferdighetsstrengene. Studieprogramlederne systematiserer data og skriver et kort sammendrag. Denne kartleggingen ble ferdigstilt primo september 2020.
Den eksterne arbeidsgruppa baserer sin evaluering basert på den interne kartleggingen. En studentgruppe vil arbeid parallelt med den eksterne komiteen med samme tema. Om ønskelig kan den eksterne arbeidsgruppen arrangere et møte med studentgruppen for å få studentenes perspektiv på innholdet av beregningsorientering og eksperimentell fysikk i utdanningene. Det anbefales å gjøre dette møtet digitalt.
Hvert medlem i arbeidsgruppa skriver en egen vurdering av innholdet av ferdighetsstrengene innenfor sitt tema. Denne vurdering er basert på dokumentasjon fra den interne kartleggingen og skal besvare spørsmålene stilt over. Den interne kartleggingen inneholder også en del konkrete spørsmål som ønskes besvart (se kartleggingsdokument).
I tillegg skal leder for arbeidsgruppa utarbeide ett sammendrag av arbeidsgruppas vurderinger, inkludert studentenes perspektiv på utdanningen. 
Frist for endelig rapport er 15. desember 2020.
Arbeidsgruppa forventes å ha minst ett møte før sammendrag og endelig rapport sammenstilles.
Arbeidsgruppa forventes å delta på et seminar i Trondheim for å legge frem arbeidsgruppas rapport og diskutere funn. Dette vil gjennomføres i løpet av januar 2021.

\subsection{Ressursbruk}
Medlemmer i arbeidsgruppa honoreres etter gjeldende satser.
Utgifter til angitte reiser og opphold dekkes.

\bibliographystyle{unsrt}
\bibliography{References}

\end{document}
